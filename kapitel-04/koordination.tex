\section{Koordination, Konsens und Fehler}

In der heutigen vernetzten Welt spielen verteilte Systeme eine entscheidende Rolle bei der Unterstützung von Anwendungen und Diensten, die von Unternehmen und Endbenutzern eingesetzt werden. Die Koordination in verteilten Systemen ist ein zentrales Thema, das es ermöglicht, die Funktionsweise dieser Systeme effektiv und effizient zu verwalten. In dieser Einführung werden wir uns auf die grundlegenden Prinzipien der Koordination konzentrieren.

Ein verteiltes System ist ein Netzwerk von autonomen Computern oder Knoten, die zusammenarbeiten, um einen gemeinsamen Zweck zu erfüllen. Diese Knoten kommunizieren über Nachrichten und teilen Ressourcen, um Aufgaben gemeinsam zu erledigen. Die Koordination in verteilten Systemen bezieht sich auf den Prozess, bei dem verschiedene Komponenten des Systems zusammenarbeiten und Informationen austauschen, um ein konsistentes und kohärentes Verhalten zu erreichen. Das wir das kohärente Verhalten durch die Erreichung der Transparenzeigenschaften erhalten können ist uns bekannt, wie wir den Begriff der Konsistenz zu verstehen haben ist noch zu diskutieren. 

Eine der Hauptanforderungen an die Koordination in verteilten Systemen ist die Synchronisation. Die Synchronisation bezieht sich auf die Notwendigkeit, die Aktivitäten der verschiedenen Knoten im System aufeinander abzustimmen, um korrekte und vorhersehbare Ergebnisse zu erzielen. Hierfür gibt es verschiedene Ansätze, beispielsweise der Einsatz von globalen Uhren, logischen Uhren oder Vektoruhren, um die Reihenfolge der Ereignisse im System zu bestimmen und Konsistenzbedingungen zu erfüllen.

Ein weiterer wichtiger Aspekt der Koordination ist die Einigung (Konsens). In verteilten Systemen müssen Knoten oft auf ein gemeinsames Ergebnis oder eine Entscheidung einigen. Dies kann beispielsweise bei der Wahl eines Leaders oder der Bestätigung einer Transaktion erforderlich sein. Es gibt verschiedene Konsensalgorithmen, wie zum Beispiel Paxos, oder Raft, die sicherstellen, dass ein solcher Konsens auch in Anwesenheit von Fehlern und Verzögerungen im System erreicht wird.

Die Handhabung von Fehlern und Wiederherstellung ist ein weiterer entscheidender Aspekt der Koordination in verteilten Systemen. Knoten können aufgrund von Hardware-, Software- oder Netzwerkfehlern ausfallen, und das System muss in der Lage sein, solche Ausfälle zu erkennen und angemessen darauf zu reagieren. Wiederherstellungsstrategien wie Replikation, Sharding und Checkpointing helfen dabei, die Datenintegrität aufrechtzuerhalten und das System nach einem Ausfall wieder in einen konsistenten Zustand zu bringen.

Eines der bekanntesten Beispiele für verteilte Systeme und ihre Koordination ist das Hadoop-Ökosystem. Hadoop ist ein Open-Source-Framework, das auf verteilten Systemen zum Speichern und Verarbeiten großer Datenmengen ausgelegt ist. Hadoop verwendet eine Master-Slave-Architektur, bei der der Hadoop Distributed File System (HDFS) NameNode die Metadaten speichert und die DataNodes die eigentlichen Daten speichern und verarbeiten. Koordination wird durch den YARN (Yet Another Resource Negotiator) erreicht, der als Cluster-Ressourcenmanager fungiert und die Ressourcen effizient auf die verschiedenen Anwendungen verteilt. In verteilten Datenbanken müssen Transaktionen, die auf unterschiedlichen Knoten ausgeführt werden, koordiniert werden, um die ACID-Eigenschaften (Atomicity, Consistency, Isolation und Durability) sicherzustellen. Hierzu kommen Techniken wie Zwei-Phasen-Commit oder Three-Phasen-Commit zum Einsatz, die eine globale Einigung über das Ergebnis einer Transaktion ermöglichen und so die Datenintegrität gewährleisten.

In der Welt des Internets ist die Domain Name System (DNS)-Hierarchie ein weiteres Beispiel für ein verteiltes System, das Koordination erfordert. DNS ist ein hierarchisches und verteilt angelegtes System zur Zuordnung von Domainnamen zu IP-Adressen. Hierarchische Strukturen wie Root-, Top-Level- und Second-Level-Domain-Server arbeiten zusammen, um die Anfragen der Benutzer effizient aufzulösen und die korrekten IP-Adressen zurückzugeben.

Insgesamt ist die Koordination in verteilten Systemen eine komplexe und essenzielle Aufgabe, um die korrekte Funktion und Leistung des Systems sicherzustellen. 

\subsection{Konsistenz}

Konsistenz ist ein wichtiges Konzept in verteilten Systemen und bezieht sich darauf, wie Datenänderungen über verschiedene Knoten hinweg sichtbar sind. In einfachen Worten bedeutet Konsistenz, dass alle Knoten im verteilten System immer die gleiche Sicht auf die Daten haben sollten. Es gibt verschiedene Stufen der Konsistenz, von strikter Konsistenz bis hin zu schwacher oder sich annährende Konsistenz, je nach den Anforderungen der Anwendung und den zugrunde liegenden Mechanismen, die zur Gewährleistung der Konsistenz eingesetzt werden.

\textbf{Strikte Konsistenz} ist das stärkste Konsistenzmodell, bei dem alle Knoten im verteilten System sofort die gleichen Daten sehen, sobald eine Änderung vorgenommen wurde. Das bedeutet, wenn ein Knoten eine Aktualisierung vornimmt, sind alle anderen Knoten sofort über diese Aktualisierung informiert, bevor sie weitere Lese- oder Schreiboperationen durchführen. In der Praxis ist es jedoch schwierig, strikte Konsistenz in verteilten Systemen zu erreichen, insbesondere aufgrund der physikalischen Begrenzungen der Informationsübertragung.

Licht ist das schnellste Medium, mit dem Informationen übertragen werden können, aber selbst Licht hat eine endliche Geschwindigkeit von etwa 299.792 Kilometern pro Sekunde (km/s) im Vakuum. Diese Geschwindigkeit begrenzt, wie schnell Informationen zwischen verschiedenen Knoten in einem verteilten System ausgetauscht werden können.

Um zu zeigen, dass selbst Licht zu langsam ist, um strikte Konsistenz zu erreichen, betrachten wir ein hypothetisches verteiltes System mit zwei Knoten, die 3.000 Kilometer voneinander entfernt sind. Um die strikte Konsistenz aufrechtzuerhalten, müssen Informationen zwischen diesen Knoten sofort ausgetauscht werden, sobald eine Änderung vorgenommen wurde.

Die minimale Zeit, die benötigt wird, um Informationen über diese Entfernung mit Lichtgeschwindigkeit zu übertragen, kann berechnet werden, indem man die Entfernung durch die Lichtgeschwindigkeit teilt:
\begin{lstlisting}[caption={Berechnung Lichtgeschwindigkeit},captionpos=b,label={lst:licht}]
t = Distanz / Lichtgeschwindigkeit
t = 3.000 km / 299.792 km/s
t =  0,01 s
\end{lstlisting}
In diesem Beispiel würde es also mindestens 0,01 Sekunden dauern, um Informationen über die Datenaktualisierung zwischen den beiden Knoten auszutauschen. Dies ist eine Ewigkeit im Kontext einer Datenbankanwendung, in der wir Lokal in einem Takt von mehreren GHz auf diesen Daten Operationen ausführen. In dieser Zeit wäre das verteilte System inkonsistent, da der zweite Knoten noch keine Kenntnis von der Aktualisierung hätte. Daher ist es praktisch unmöglich, strikte Konsistenz in verteilten Systemen zu erreichen, selbst wenn die Informationsübertragung mit Lichtgeschwindigkeit stattfindet.

In realen verteilten Systemen sind Konsistenzmodelle wie sequentielle Konsistenz, kausale Konsistenz oder sich annährende Konsistenz häufiger anzutreffen, da sie ein ausgewogenes Verhältnis zwischen Leistung und Konsistenz bieten und besser mit den inhärenten Latenzen und Unwägbarkeiten in verteilten Systemen umgehen können.

Neben physikalischen Größen hält aber bereits die Logik eine Herausforderung für die Konsistenz  bereit. 

\subsection{CAP Theorem}
Das CAP-Theorem, auch bekannt als Brewer's Theorem, ist ein fundamentales Prinzip, das die inhärenten Trade-offs in verteilten Systemen beschreibt. Es besagt, dass ein verteiltes System nur zwei der folgenden drei Eigenschaften gleichzeitig erfüllen kann: Konsistenz (C), Verfügbarkeit (A) und Partitionstoleranz (P).
\begin{itemize}
\item \textbf{Konsistenz} (C): Alle Knoten im System sehen zur gleichen Zeit dieselben Daten.
\item \textbf{Verfügbarkeit} (A): Alle Knoten sind stets in der Lage, Anfragen von Clients zu bearbeiten und korrekte Antworten zu liefern.
\item \textbf{Partitionstoleranz} (P): Das System kann trotz Netzwerkpartitionen, bei denen die Kommunikation zwischen Knoten unterbrochen ist, weiterhin korrekt funktionieren.
\end{itemize}

Das CAP-Theorem impliziert, dass verteilte Systeme in der Praxis immer zwischen Konsistenz und Verfügbarkeit abwägen müssen, da Partitionstoleranz als grundlegende Anforderung für verteilte Systeme angesehen wird.

Der Grund für diesen Trade-off liegt in den Unwägbarkeiten und Latenzen, die in verteilten Systemen vorhanden sind. Netzwerkpartitionen können aufgrund von Hardware-, Software- oder Konfigurationsfehlern auftreten. Wenn ein System auf strikte Konsistenz besteht, kann es während einer Netzwerkpartition nicht verfügbar sein, da es nicht möglich ist, alle Knoten zu synchronisieren, bis die Partition behoben ist. Andererseits, wenn ein System auf hohe Verfügbarkeit besteht, müssen Anfragen auch während einer Netzwerkpartition beantwortet werden, selbst wenn dies bedeutet, dass die Konsistenz beeinträchtigt wird.
\\\\
In realen Anwendungen ist der Trade-off zwischen Konsistenz und Verfügbarkeit häufig von den spezifischen Anforderungen der Anwendung abhängig. Zum Beispiel sind in einigen Anwendungen, wie Finanztransaktionssystemen, Konsistenz und Datenintegrität von größter Bedeutung. In solchen Fällen wäre ein System, das stärker auf Konsistenz ausgerichtet ist, wie beispielsweise ein System, das auf dem Paxos- oder Raft-Konsensalgorithmus basiert, besser geeignet.

In anderen Anwendungen, wie beispielsweise sozialen Netzwerken oder Content-Delivery-Netzwerken, ist Verfügbarkeit möglicherweise wichtiger als strenge Konsistenz. In solchen Fällen können Systeme, die auf schwacher Konsistenz oder  sich annährende Konsistenz basieren, wie beispielsweise Amazon's Dynamo oder Apache Cassandra, besser geeignet sein, um den Anforderungen gerecht zu werden.

Insgesamt zeigt das CAP-Theorem die fundamentale Herausforderung bei der Gestaltung von verteilten Systemen auf und betont die Notwendigkeit, den Trade-off zwischen Konsistenz und Verfügbarkeit in Abhängigkeit von den spezifischen Anforderungen und Prioritäten der Anwendung sorgfältig abzuwägen.

\subsection{Konsistenzmodell}

Da verteilte Systeme immer häufiger in der modernen Computertechnik eingesetzt werden, ist es wichtig, fortgeschrittene Techniken und Ansätze zu entwickeln, um den Konsistenz-Verfügbarkeits-Trade-off effektiv zu bewältigen. Einige solcher Ansätze umfassen:
\begin{itemize}
\item Tunable Consistency: Eine Möglichkeit, den Konsistenz-Verfügbarkeits-Trade-off zu bewältigen, besteht darin, Systeme mit anpassbarer Konsistenz (tunable consistency) zu entwerfen. In solchen Systemen kann die Konsistenzstufe für verschiedene Teile der Anwendung oder sogar für verschiedene Anfragen dynamisch angepasst werden. Ein Beispiel hierfür ist Apache Cassandra, bei dem die Konsistenzstufe für Lese- und Schreiboperationen auf einer Per-Anfrage-Basis angepasst werden kann.
\item Quorum-Systeme: Quorum-Systeme sind ein weiterer Ansatz zur Bewältigung des Konsistenz-Verfügbarkeits-Trade-offs. In einem Quorum-System wird die Mehrheit (oder ein festgelegter Prozentsatz) der Knoten benötigt, um eine Entscheidung zu treffen oder eine Operation durchzuführen. Dies ermöglicht eine gewisse Konsistenz und Verfügbarkeit, auch in Anwesenheit von Netzwerkpartitionen. Beispiele für Quorum-basierte Systeme sind Google Spanner und Amazon DynamoDB.
\item CRDTs (Conflict-free Replicated Data Types): CRDTs sind spezielle Datenstrukturen, die entwickelt wurden, um in verteilten Systemen repliziert zu werden, ohne dass es zu Konflikten kommt. Sie ermöglichen das Zusammenführen von Updates aus verschiedenen Knoten, ohne dass ein zentraler Koordinator erforderlich ist. CRDTs können dazu beitragen, die Verfügbarkeit zu erhöhen, während sie gleichzeitig eine gewisse Konsistenz aufrechterhalten. Ein Beispiel für ein System, das CRDTs verwendet, ist Riak.
\item Hybride Ansätze: In einigen Fällen kann es sinnvoll sein, hybride Ansätze zu verwenden, die verschiedene Konsistenzmodelle und Techniken kombinieren, um den Konsistenz-Verfügbarkeits-Trade-off effektiv zu bewältigen. Ein Beispiel hierfür ist das COPS-System (Clusters of Order-Preserving Servers), das eine Kombination aus sequentieller Konsistenz und eventual Konsistenz verwendet, um Daten über mehrere Rechenzentren hinweg zu replizieren.
\end{itemize}
Mit welchen Mitteln man sich auch immer hilft, wichtig ist das/ein Konsistenzmodell zu verstehen, dass die Freiheitsgrade deutlich macht in dem man sich bewegen kann. Konsistenzmodelle beschreiben, wie ein verteiltes System den Zustand seiner Daten unter den verschiedenen Knoten synchronisiert und aufrechterhält. 
\\\\
Es gibt verschiedene Konsistenzmodell, wie jenes, das von Kyle Kingsbury entwickelt wurde. Das Modell ist allerdings  kein spezifisches Konsistenzmodell, sondern ein Framework zur Analyse und Validierung von verteilten Systemen hinsichtlich ihrer Datenkonsistenz- und Sicherheitseigenschaften, das verschiedene Modelle aufnimmt und sie ordnet. Diese Ordnung ist nicht standardisiert, so dass es in der Praxis Überlappungen und unterschiedliche Interpretationen geben kann. 
\\\\
In dieser Ausarbeitung konzentriert sich das Modellkonzept auf die Unterscheidung von Datenzentrische Konsistenzmodelle und Client-zentrische Konsistenzmodelle. Datenzentrische Konsistenzmodelle konzentrieren sich darauf, wie verteilte Systeme den Zustand ihrer Daten unter den verschiedenen Knoten synchronisieren und aufrechterhalten. Sie beschreiben die Konsistenzanforderungen auf Systemebene und betreffen die Sichtbarkeit und Reihenfolge von Lese- und Schreiboperationen auf den Knoten.
\\\\
Client-zentrische Konsistenzmodelle hingegen legen den Fokus auf die Sichtbarkeit und Reihenfolge von Lese- und Schreiboperationen aus der Perspektive eines einzelnen Clients. Sie beschreiben, welche Garantien ein verteiltes System in Bezug auf die Konsistenz für einen bestimmten Client oder eine Gruppe von Clients bietet.
\\\\
Kurz gesagt, datenzentrische Konsistenzmodelle beziehen sich auf die Konsistenz auf der Ebene des verteilten Systems, während client-zentrische Konsistenzmodelle die Konsistenz aus der Perspektive der Clients betrachten. Zunächst konzentriert sich das Skript auf die datenzentrische Sicht. 

\subsection{Datenzentrische Konsistenzmodelle}
Einige der datenzentrischen Konsistenzmodelle, die im Rahmen der Diskussion analysiert werden sollten, sind:
\begin{itemize}
\item \textbf{Strong Consistency}: Eine verteilte Datenbank stellt sicher, dass alle Operationen auf allen Knoten gleichzeitig abgeschlossen werden. Diese Art von Konsistenzmodell stellt sicher, dass jeder Benutzer immer die neuesten Daten sieht.
\item \textbf{Linearizability}: Linearizability stellt sicher, dass alle Operationen auf allen Knoten in einer einzigen, atomaren Reihenfolge ausgeführt werden, die den Echtzeit-Verhalten entspricht.
\item \textbf{Sequential Consistency}: In diesem Modell müssen alle Operationen in einer einzigen, globalen Reihenfolge ausgeführt werden, die für alle Knoten gleich ist. Dieses Modell stellt jedoch keine kausale Beziehung zwischen Operationen her.
\item Causal Consistency: Causal Consistency stellt sicher, dass alle Operationen, die kausal voneinander abhängig sind, in der gleichen Reihenfolge auf allen Knoten angewendet werden. Unabhängige Operationen können jedoch in unterschiedlicher Reihenfolge ausgeführt werden.
\item \textbf{Eventual Consistency}: In diesem Modell werden Änderungen an Daten auf verschiedenen Knoten schließlich synchronisiert, aber nicht sofort. Dadurch kann es zu vorübergehenden Inkonsistenzen kommen, die jedoch im Laufe der Zeit behoben werden.
\end{itemize}

In diesem Skript wird versucht sich mit systematischen Methoden, um die Konsistenz- und Sicherheitseigenschaften von verteilten Systemen zu nähern, um sie entsprechend dem Anwendungsfall analysieren und sicherzustellen zu können. Da wir bereits diskutiert haben das Strong Consistency in einem verteilten System kaum zu erreichen ist, konzentriert man sich auf Modelle mit schwächerer Konsistenz, somit wäre das nächste in der Liste Linearizability. 

\subsubsection{Atomare Konsistenz}

Linearizability, auch als atomare Konsistenz bezeichnet oder manchmal mit starker (strong) Konsistenz gleichgesetzt, ist ein strenges Konsistenzmodell, bei dem jede Operation in einem verteilten System so aussieht, als ob sie sofort und in einer atomaren Weise auf alle Knoten im System angewendet wird. Linearizability stellt sicher, dass es eine globale Reihenfolge für alle Operationen gibt und dass diese Reihenfolge den Echtzeit-Verlauf der Operationen widerspiegelt.

In der Praxis kann die Umsetzung von Linearizability in verteilten Systemen eine Herausforderung darstellen, da sie strenge Anforderungen an die Synchronisation und Kommunikation zwischen Knoten stellt. Hier sind einige Faktoren, die bei der praktischen Umsetzung von Linearizability berücksichtigt werden müssen:

\begin{itemize}
\item Latenz: Da Linearizability erfordert, dass alle Knoten im System synchronisiert werden, bevor eine Operation als abgeschlossen betrachtet werden kann, kann dies zu erhöhter Latenz bei Lese- und Schreiboperationen führen. Insbesondere in geografisch verteilten Systemen kann die Netzwerklatenz erhebliche Auswirkungen auf die Leistung haben.
\item Durchsatz: Die Anforderungen an die Synchronisation und Kommunikation zwischen Knoten, die für die Umsetzung von Linearizability erforderlich sind, können auch den Durchsatz des Systems beeinträchtigen, da Knoten möglicherweise auf die Bestätigung anderer Knoten warten müssen, bevor sie weitere Operationen durchführen können.
\item Verfügbarkeit: In Anbetracht des CAP-Theorems kann die Umsetzung von Linearizability die Verfügbarkeit des Systems beeinträchtigen, insbesondere in Szenarien, in denen Netzwerkpartitionen auftreten.
\end{itemize}
Um Linearizability in verteilten Systemen praktisch umzusetzen, werden häufig Konsensalgorithmen wie Paxos, Raft oder Zab verwendet. Diese Protokolle führen die Idee eines Zyklus ein, welches das System Synchronisiert und Konsens etabliert. Die Einhaltungen der Anforderungen im Zyklus können herausfordernd sein, und nicht alle Abweichungen im Verhalten können einer Lösung zugeführt werden. Dennoch, diese Algorithmen ermöglichen es den Knoten, sich auf eine gemeinsame Reihenfolge von Operationen zu einigen und sicherzustellen, dass die Linearizability-Anforderungen erfüllt sind. Diese Algorithmen werden im späteren Kapitel nochmals genauer betrachtet. Trotz der Herausforderungen bei der Umsetzung von Linearizability kann dieses Konsistenzmodell in bestimmten Anwendungsfällen, in denen Datenintegrität und Konsistenz von größter Bedeutung sind (z. B. in Finanztransaktionssystemen), von Vorteil sein. Dennoch müssen Systemarchitekten und Entwickler die Auswirkungen auf Latenz, Durchsatz und Verfügbarkeit sorgfältig abwägen und möglicherweise weniger strenge Konsistenzmodelle in Betracht ziehen, wenn die Anforderungen der Anwendung dies zulassen.

\subsubsection{Sequentielle Konsistenz}
Sequential Consistency, oder sequentielle Konsistenz, ist ein Konsistenzmodell, bei dem die Operationen zum Beispiel in einer verteilten Datenbank in einer Reihenfolge ausgeführt werden, die für alle Knoten gleich ist. Während die tatsächliche Synchronisation der Knoten verzögert auftreten kann, bleibt die Reihenfolge der Operationen über alle Knoten hinweg konsistent.
\\\\
Um dies zu veranschaulichen, betrachten wir ein verteiltes System mit drei Knoten A, B und C, das einen gemeinsamen Zähler speichert. Nehmen wir an, es gibt zwei Benutzer, Benutzer 1 und Benutzer 2, die gleichzeitig den Zähler inkrementieren möchten. Der anfängliche Wert des Zählers ist 0.

Benutzer 1 sendet eine Inkrementanforderung an Knoten A, während Benutzer 2 eine Inkrementanforderung an Knoten B sendet. Da die beiden Anforderungen gleichzeitig erfolgen, ist die Reihenfolge der Operationen unklar. Um sequentielle Konsistenz zu gewährleisten, müssen alle Knoten jedoch eine gemeinsame Reihenfolge der Operationen einhalten.
\\\\
Angenommen, das verteilte System einigt sich darauf, dass die Operation von Benutzer 1 vor der Operation von Benutzer 2 ausgeführt wird. In diesem Fall müssen alle Knoten die Inkrementanforderung von Benutzer 1 zuerst verarbeiten und den Zähler auf 1 erhöhen. Anschließend verarbeiten sie die Inkrementanforderung von Benutzer 2 und erhöhen den Zähler auf 2.
Diese Vereinbarung kann auf vielfältige Weise getroffen werden, wie zum Beispiel vordefiniert oder über ein Abstimmungsverhalten. 
\\\\
Wichtig ist, dass die Reihenfolge der Operationen auf allen Knoten gleich ist, auch wenn die tatsächliche Synchronisation der Knoten verzögert erfolgen kann. Zum Beispiel kann Knoten C die Inkrementanforderungen von Benutzer 1 und Benutzer 2 etwas später als Knoten A und B erhalten. Solange Knoten C jedoch die Operationen in derselben Reihenfolge verarbeitet, wird die sequentielle Konsistenz eingehalten.
\\\\
Dieses Konsistenzmodell bietet eine ausgewogene Mischung aus Konsistenz und Leistung und wird häufig in verteilten Systemen verwendet, bei denen eine strikte oder kausale Konsistenz nicht unbedingt erforderlich ist.


\subsubsection{Kausale Konsistenz}
Kausale Konsistenz ist ein Konsistenzmodell in verteilten Systemen, das darauf abzielt, die kausalen Beziehungen zwischen Operationen zu erhalten. Im Wesentlichen bedeutet dies, dass Operationen, die kausal voneinander abhängig sind, in der gleichen Reihenfolge auf allen Knoten im verteilten System ausgeführt werden müssen. Im Gegensatz dazu können Operationen, die keine kausale Beziehung zueinander haben, in unterschiedlicher Reihenfolge auf verschiedenen Knoten ausgeführt werden, ohne die Konsistenz zu beeinträchtigen.
\\\\
Ein praktisches Beispiel zur Veranschaulichung der kausalen Konsistenz wäre ein verteiltes soziales Netzwerk, bei dem Benutzer Nachrichten posten und auf Nachrichten anderer Benutzer antworten können. Die kausale Konsistenz stellt sicher, dass Antworten in der richtigen Reihenfolge angezeigt werden, bezogen auf die zugrunde liegenden Nachrichten.

Betrachten Sie folgendes Szenario: Alice und Bob sind Benutzer des sozialen Netzwerks und haben Zugriff auf verschiedene Knoten des verteilten Systems. Alice postet eine Nachricht (Nachricht A) und Bob sieht diese Nachricht und antwortet darauf (Antwort B). In diesem Fall gibt es eine kausale Beziehung zwischen Nachricht A und Antwort B, da Antwort B eine direkte Reaktion auf Nachricht A ist.

Da kausale Konsistenz die kausalen Beziehungen zwischen Operationen erhalten muss, muss Antwort B in der richtigen Reihenfolge nach Nachricht A auf allen Knoten im verteilten System angezeigt werden. Unabhängig von der tatsächlichen Zeit, zu der die Knoten die Informationen über die Nachricht A und Antwort B erhalten, stellt die kausale Konsistenz sicher, dass die Antwort B niemals vor der Nachricht A angezeigt wird.

Angenommen, ein anderer Benutzer, Carol, postet währenddessen eine unabhängige Nachricht (Nachricht C), die keine Beziehung zu den Nachrichten von Alice oder Bob hat. Da es keine kausale Beziehung zwischen Nachricht C und den anderen Nachrichten gibt, kann Nachricht C in einer beliebigen Reihenfolge in Bezug auf Nachricht A und Antwort B auf verschiedenen Knoten erscheinen, ohne die kausale Konsistenz zu beeinträchtigen.
\\\\
In diesem Beispiel haben wir gesehen, wie kausale Konsistenz dazu beiträgt, die kausalen Beziehungen zwischen Operationen in verteilten Systemen zu erhalten, indem sie sicherstellt, dass kausal abhängige Operationen in der richtigen Reihenfolge auf allen Knoten ausgeführt werden. Kausale Konsistenz ist besonders nützlich in Anwendungen, bei denen die zeitliche Abfolge von Ereignissen wichtig ist, wie beispielsweise in sozialen Netzwerken, Foren oder Kommunikationssystemen.
\\\\
Kausale Konsistenz bietet eine nützliche Balance zwischen Leistung und Konsistenz in verteilten Systemen, indem sie eine natürliche Ordnung für kausal verwandte Operationen aufrechterhält, ohne die Systemleistung übermäßig zu beeinträchtigen. Um kausale Konsistenz in der Praxis umzusetzen, werden häufig Vektoruhren oder andere Techniken zur Erfassung von kausalen Beziehungen eingesetzt.

Ein weiterer Ansatz zur Implementierung kausaler Konsistenz ist die Verwendung von Version Vektoren. Sie ähneln Vektoruhren, erfassen jedoch die Versionen von Objekten anstelle von Ereignissen. Version Vektoren ermöglichen es, Konflikte zwischen kausalen Versionen von Objekten zu erkennen und aufzulösen.

\subsubsection{Gelegentliche Konsistent}
Eventual Consistency ist ein Konsistenzmodell für verteilte Systeme, bei dem die Daten im System möglicherweise für eine gewisse Zeit inkonsistent sein können, aber schließlich eine konsistente Zustand erreichen, wenn keine weiteren Updates mehr erfolgen. Dieses Modell bietet eine höhere Verfügbarkeit und Skalierbarkeit, indem es vorübergehende Inkonsistenzen in Kauf nimmt und eine lockere Konsistenz zwischen den Knoten des Systems zulässt.

In einem verteilten System, das Eventual Consistency implementiert, kann es vorkommen, dass verschiedene Knoten unterschiedliche Versionen von Daten haben, insbesondere wenn gleichzeitige Updates stattfinden oder die Kommunikation zwischen den Knoten verzögert ist. Im Laufe der Zeit werden jedoch die Knoten die Updates replizieren und die Daten synchronisieren, bis alle Knoten eine konsistente Ansicht der Daten haben.

Ein praktisches Beispiel für Eventual Consistency ist das Amazon DynamoDB-System, ein hochverfügbarer und skalierbarer NoSQL-Datenbankservice. Amazon DynamoDB verwendet Eventual Consistency, um eine hohe Verfügbarkeit und Leistung zu gewährleisten, selbst wenn das System über mehrere Rechenzentren und geografische Standorte verteilt ist.

Angenommen, ein E-Commerce-Unternehmen verwendet Amazon DynamoDB, um die Bestandsinformationen für seine Produkte zu verwalten. Wenn ein Kunde eine Bestellung aufgibt und ein Produkt kauft, wird der Bestand des Produkts in DynamoDB aktualisiert. Aufgrund der Eventual Consistency kann es jedoch vorkommen, dass verschiedene Knoten im DynamoDB-System unterschiedliche Bestandszahlen für das Produkt haben, während die Updates repliziert werden.

Ein weiterer Kunde, der zur gleichen Zeit die Produktseite besucht, sieht möglicherweise die alte Bestandszahl, die noch nicht aktualisiert wurde. Im Laufe der Zeit wird jedoch das Update auf alle Knoten im System repliziert, und alle Kunden werden schließlich die korrekte und konsistente Bestandszahl sehen.

Dieses Beispiel verdeutlicht das Konzept der Eventual Consistency, bei dem verteilte Systeme vorübergehende Inkonsistenzen zulassen, um eine höhere Verfügbarkeit und Skalierbarkeit zu erzielen, und schließlich eine konsistente Ansicht der Daten erreichen, wenn keine weiteren Updates mehr erfolgen.

Eventual Consistency ist für Entwickler von großer Bedeutung, insbesondere wenn sie verteilte Systeme oder Anwendungen mit hohen Verfügbarkeits- und Skalierbarkeitsanforderungen entwerfen. Die Implementierung von Eventual Consistency in einer Anwendung kann dazu beitragen, einige Herausforderungen im Zusammenhang mit verteilten Systemen zu bewältigen.


\subsection{Client-zentrische Konsistenzmodelle}
Client-zentrische Konsistenzmodelle sind eine Kategorie von Konsistenzmodellen in verteilten Systemen, die sich darauf konzentrieren, die Konsistenz aus der Perspektive der Clients oder Benutzer des Systems aufrechtzuerhalten. Im Gegensatz zu datenzentrischen Konsistenzmodellen, die sich auf die Konsistenz der Daten innerhalb des verteilten Systems konzentrieren, stellen client-zentrische Modelle sicher, dass die Clients eine konsistente Ansicht der Daten erhalten, auch wenn die Daten im System selbst möglicherweise nicht vollständig konsistent sind.

\subsubsection{Monotonic Reads}

Monotonic Reads ist ein client-zentrisches Konsistenzmodell, das die Konsistenz der Daten aus der Perspektive eines Clients oder Benutzers gewährleistet. Bei Monotonic Reads erhält ein Client, der wiederholt Lesevorgänge auf einem verteilten System ausführt, niemals ältere Daten als die, die er zuvor gelesen hat. Sobald ein Client eine bestimmte Version der Daten gelesen hat, werden alle nachfolgenden Lesevorgänge des Clients Daten liefern, die gleich oder neuer als die zuvor gelesene Version sind. Dieses Modell verhindert, dass ein Client Inkonsistenzen aufgrund von veralteten Daten bemerkt.

Ein konkretes Beispiel für Monotonic Reads ist ein verteiltes E-Commerce-System, bei dem Benutzer den Lagerbestand von Artikeln abfragen können. Angenommen, ein Benutzer fragt den Lagerbestand eines bestimmten Artikels ab und erhält die Antwort, dass 10 Einheiten auf Lager sind. Kurz darauf wird der Lagerbestand aufgrund einer neuen Lieferung auf 15 Einheiten erhöht. Wenn der Benutzer später erneut den Lagerbestand abfragt, sollte das System nach dem Monotonic Reads-Prinzip sicherstellen, dass der Benutzer mindestens 15 Einheiten als Antwort erhält und nicht die ältere Information von 10 Einheiten, die er zuvor gesehen hat.

In der Praxis kann die Umsetzung von Monotonic Reads in einem verteilten System verschiedene Ansätze erfordern, abhängig von der Architektur und den Anforderungen des Systems. Eine mögliche Methode zur Implementierung von Monotonic Reads besteht darin, einen Zeitstempel oder eine Versionsnummer für jeden Datensatz zu speichern, die bei jeder Aktualisierung des Datensatzes erhöht wird. Wenn ein Client eine Leseanfrage stellt, kann das System den Zeitstempel oder die Versionsnummer der zuletzt gelesenen Version speichern und sicherstellen, dass alle nachfolgenden Leseanfragen nur Daten zurückgeben, die gleich oder neuer als dieser Zeitstempel oder diese Versionsnummer sind.

Ein praktischer Einsatz von Monotonic Reads könnte beispielsweise in einem verteilten Datenbanksystem oder einer Anwendung zur Verwaltung verteilter Dateisysteme erfolgen. In solchen Systemen ist es wichtig, dass Clients eine konsistente Ansicht der Daten erhalten, um Inkonsistenzen oder Verwirrung bei den Benutzern zu vermeiden. Durch die Implementierung von Monotonic Reads kann das System sicherstellen, dass die Benutzer immer eine konsistente und fortschreitende Ansicht der Daten erhalten, selbst wenn das System selbst mit Replikations- oder Synchronisationsherausforderungen konfrontiert ist.

\subsubsection{Monotonic Writes}

Monotonic Writes ist ein client-zentrisches Konsistenzmodell, das darauf abzielt, die Konsistenz der Daten aus der Perspektive eines Clients oder Benutzers zu gewährleisten, indem sichergestellt wird, dass die Schreibvorgänge eines Clients in der Reihenfolge ausgeführt werden, in der sie vom Client eingereicht wurden. Im Gegensatz zu Monotonic Reads, bei dem es um die Konsistenz von Lesevorgängen geht, konzentriert sich Monotonic Writes darauf, die richtige Reihenfolge von Schreibvorgängen beizubehalten, um Inkonsistenzen zu vermeiden, die durch ungeordnete Schreibvorgänge entstehen können.

Ein Fallbeispiel für Monotonic Writes ist ein verteiltes System zur Verwaltung von Banktransaktionen. Angenommen, ein Benutzer führt zwei Überweisungen in einer bestimmten Reihenfolge aus: zunächst eine Überweisung von 100 € von Konto A auf Konto B und anschließend eine Überweisung von 50 € von Konto B auf Konto C. Monotonic Writes stellt sicher, dass diese Transaktionen in der korrekten Reihenfolge verarbeitet werden, um Konsistenzprobleme zu vermeiden, die entstehen könnten, wenn die zweite Transaktion vor der ersten Transaktion verarbeitet würde.

In der Praxis kann die Implementierung von Monotonic Writes in einem verteilten System verschiedene Ansätze erfordern, abhängig von der Systemarchitektur und den Anforderungen. Eine mögliche Methode zur Umsetzung von Monotonic Writes besteht darin, die Schreibvorgänge mit Zeitstempeln oder Versionsnummern zu versehen und die Schreibvorgänge in einer Warteschlange zu halten, bis sie in der richtigen Reihenfolge verarbeitet werden können. Eine weitere Möglichkeit besteht darin, ein verteiltes Transaktionsprotokoll zu verwenden, um die Reihenfolge der Schreibvorgänge zu verfolgen und sicherzustellen, dass sie in der richtigen Reihenfolge ausgeführt werden.

\subsubsection{Read your Writes}
Read Your Writes ist ein client-zentrisches Konsistenzmodell, das darauf abzielt, die Konsistenz der Daten aus der Perspektive eines Clients oder Benutzers zu gewährleisten, indem sichergestellt wird, dass ein Client nach dem Ausführen eines Schreibvorgangs immer die von ihm geschriebenen Daten lesen kann. Dies stellt sicher, dass der Client die Auswirkungen seiner eigenen Schreibvorgänge sofort sieht und eine konsistente Ansicht der Daten erhält.

Ein Fallbeispiel für Read Your Writes ist ein verteiltes System für soziale Netzwerke, bei dem Benutzer Statusaktualisierungen veröffentlichen können. Angenommen, ein Benutzer veröffentlicht eine neue Statusaktualisierung. Nach dem Veröffentlichen der Aktualisierung möchte der Benutzer die Liste seiner eigenen Statusaktualisierungen anzeigen. Read Your Writes stellt sicher, dass der Benutzer seine neueste Aktualisierung in der Liste sieht, selbst wenn das verteilte System noch dabei ist, die Aktualisierung auf alle Knoten zu replizieren.

In der Praxis kann die Implementierung von Read Your Writes in einem verteilten System verschiedene Ansätze erfordern, abhängig von der Architektur und den Anforderungen des Systems. Eine mögliche Methode zur Umsetzung von Read Your Writes besteht darin, einen Cache auf Clientseite zu verwenden, um die kürzlich geschriebenen Daten zu speichern, sodass der Client die eigenen Schreibvorgänge sofort lesen kann, selbst wenn sie noch nicht auf alle Knoten im verteilten System repliziert wurden. Eine weitere Möglichkeit besteht darin, ein verteiltes Transaktionsprotokoll zu verwenden, um sicherzustellen, dass alle Lesevorgänge, die nach einem Schreibvorgang ausgeführt werden, die neuesten Daten enthalten.

\subsubsection{Writes follow Reads}
Kausale Konsistenz und das \enquote{Writes follow Reads}-Prinzip sind eng miteinander verknüpft, da beide darauf abzielen, kausale Abhängigkeiten zwischen Operationen in verteilten Systemen aufrechtzuerhalten. "Writes follow Reads" ist eine Regel, die besagt, dass nach einer Leseoperation, die von einem bestimmten Knoten ausgeführt wird, alle Schreiboperationen, die von diesem Knoten initiiert werden und sich auf das gleiche Objekt beziehen, die gelesene Version oder eine neuere Version des Objekts berücksichtigen müssen.

Diese Regel hilft dabei, die kausale Ordnung der Operationen zu erhalten, indem sie sicherstellt, dass Schreiboperationen, die auf Leseoperationen basieren, in einer logisch konsistenten Reihenfolge erfolgen. Kausale Konsistenz und das \enquote{Writes follow Reads}-Prinzip ergänzen sich, da beide darauf abzielen, die kausale Reihenfolge der Operationen in verteilten Systemen zu respektieren und ein konsistentes Verhalten über verschiedene Knoten hinweg sicherzustellen.

In einem System, das kausale Konsistenz und \enquote{Writes follow Reads} implementiert, werden sowohl Schreib- als auch Leseoperationen in einer Weise ausgeführt, die ihre kausalen Abhängigkeiten berücksichtigt. Dies kann durch den Einsatz von Techniken wie Vektoruhren oder Version Vektoren erreicht werden, die dabei helfen, die kausalen Beziehungen zwischen Operationen zu erfassen und sicherzustellen, dass alle Knoten im verteilten System kausal abhängige Operationen in der richtigen Reihenfolge ausführen.

Ein reales Beispiel für die Umsetzung des \enquote{Writes follow Reads}-Prinzips ist in verteilten Datenbanken und Speichersystemen wie Apache Cassandra zu finden. 

In Cassandra basiert die Konsistenz der Daten auf einem kausalen Konsistenzmodell, das das \enquote{Writes follow Reads}-Prinzip unterstützt. Um dies zu erreichen, verwendet Cassandra Version Vektoren, sogenannte Timestamps, um kausale Beziehungen zwischen Schreib- und Leseoperationen zu erfassen und zu verwalten.

Wenn ein Benutzer in Cassandra eine Leseoperation durchführt, erhält er die neueste Version des angeforderten Objekts zusammen mit dessen Timestamp. Wenn der Benutzer daraufhin eine Schreiboperation auf demselben Objekt ausführt, muss er sicherstellen, dass der Timestamp der neuen Version höher ist als der Timestamp der gelesenen Version. Dies gewährleistet, dass Schreiboperationen auf den zuvor gelesenen Daten basieren und in einer logisch konsistenten Reihenfolge erfolgen.

Ein konkretes Beispiel dafür ist die Verwendung von Cassandra zur Speicherung und Verwaltung von Benutzerprofilen in einem sozialen Netzwerk. Angenommen, ein Benutzer liest sein Profil und ändert dann seine E-Mail-Adresse. Im Rahmen des \enquote{Writes follow Reads}-Prinzips stellt Cassandra sicher, dass die neue E-Mail-Adresse auf der gelesenen Version des Profils basiert und in der richtigen Reihenfolge gespeichert wird.

Ein weiteres Beispiel kann auch in einem Mehrbenutzersystem besprochen werden. Google Docs ist ein webbasiertes kollaboratives Textverarbeitungssystem, das es mehreren Benutzern ermöglicht, in Echtzeit an einem gemeinsamen Dokument zu arbeiten. Google Docs verwendet das Operational Transformation (OT) Framework, um die Konsistenz in Echtzeit zu gewährleisten und gleichzeitige Änderungen von verschiedenen Benutzern zu verwalten.

Angenommen, Alice und Bob arbeiten gleichzeitig an einem gemeinsamen Dokument in Google Docs. Alice liest einen Absatz und entscheidet, dass sie eine Änderung daran vornehmen möchte. Während sie ihre Änderungen vornimmt, liest Bob denselben Absatz und beschließt, ihn ebenfalls zu bearbeiten.

In diesem Szenario ist es wichtig, dass das System das \enquote{Writes follow Reads}-Prinzip einhält, um Konsistenz und kausale Ordnung der Operationen zu gewährleisten. Das OT-Framework in Google Docs stellt sicher, dass die Schreiboperationen von Alice und Bob auf den zuvor gelesenen Versionen des Absatzes basieren und in einer logisch konsistenten Reihenfolge angewendet werden.

Wenn Alice ihre Änderungen vornimmt, erfasst das System die kausale Beziehung zwischen ihrer Lese- und Schreiboperation und wendet ihre Änderung auf den zuvor gelesenen Absatz an. Gleichzeitig wird Bobs Schreiboperation ebenfalls auf der Grundlage seiner gelesenen Version des Absatzes durchgeführt. Anschließend verwendet das OT-Framework Transformationsfunktionen, um mögliche Konflikte zwischen Alices und Bobs Änderungen aufzulösen und eine konsistente Ansicht des Dokuments für beide Benutzer zu gewährleisten.

\subsection{Alternative Konsistensmodelle und Sonderformen}
Alternative Konsistenzmodelle sind Ansätze zur Datenkonsistenz in verteilten Systemen, die von den traditionellen ACID-Eigenschaften (Atomicity, Consistency, Isolation, Durability) abweichen. Diese Modelle wurden entwickelt, um die Leistung und Verfügbarkeit von verteilten Systemen zu verbessern, indem sie auf einige der Einschränkungen von ACID-basierten Ansätzen verzichten.

Einige der alternativen Konsistenzmodelle sind:
\begin{itemize}
\item BASE (Basically Available, Soft state, Eventually consistent): Dieses Modell legt den Fokus auf Verfügbarkeit und Partitionstoleranz, anstatt auf strikte Konsistenz. Es akzeptiert, dass Daten in verteilten Systemen vorübergehend inkonsistent sein können, solange sie sich langfristig wieder konsistent verhalten. Es wird oft in NoSQL-Datenbanken und Cloud-Systemen eingesetzt.
\item PACELC (Partition-tolerance, Availability, Consistency, Eventual consistency, Latency, and Consistency trade-offs): Dieses Modell berücksichtigt die trade-offs zwischen Konsistenz, Verfügbarkeit und Partitionstoleranz. Es geht davon aus, dass es unmöglich ist, alle drei Eigenschaften gleichzeitig in einem verteilten System zu erreichen, und dass eine Abwägung der Prioritäten erforderlich ist.
\end{itemize}
Diese alternativen Konsistenzmodelle bieten verschiedene Vor- und Nachteile gegenüber den traditionellen ACID-Modellen. Auch existieren Konsistenzmodelle, die keine reine Form repräsentieren sondern eher als Sonderform oder hybride Form betrachtet werden. Folgend sollen einige Beispiele eingebracht werden. 

\subsubsection{Kontinuierliche Konsistenz (hybrid)}
Continuous Consistency (kontinuierliche Konsistenz) ist ein Konsistenzmodell, das versucht, einen Kompromiss zwischen strikter Konsistenz und verfügbarkeitsorientierten Konsistenzmodellen wie eventual consistency zu finden. Anstatt die Konsistenzbedingungen strikt einzuhalten oder zu lockern, setzt Continuous Consistency auf eine kontinuierliche Verbesserung der Konsistenz im Laufe der Zeit.

Die Idee hinter Continuous Consistency besteht darin, Metriken und Schwellenwerte für die Konsistenz der Daten in einem verteilten System festzulegen und darauf hinzuarbeiten, diese Schwellenwerte im Laufe der Zeit einzuhalten oder zu überschreiten. Diese Metriken können beispielsweise die maximale Anzahl von Inkonsistenzen, die maximale Inkonsistenzdauer oder die maximale Inkonsistenzgröße umfassen.

In einem System mit kontinuierlicher Konsistenz wird die Konsistenz schrittweise verbessert, indem inkonsistente Daten erkannt und synchronisiert werden, bis die festgelegten Schwellenwerte erreicht oder überschritten sind. Dieser Ansatz ermöglicht eine bessere Balance zwischen Konsistenz und Verfügbarkeit, indem er es erlaubt, Konsistenzanforderungen basierend auf den Anforderungen der Anwendung und den zugrunde liegenden Systembedingungen flexibel anzupassen.

Continuous Consistency ist besonders nützlich für Anwendungen und Systeme, bei denen eine gewisse Toleranz gegenüber temporären Inkonsistenzen besteht, aber dennoch ein gewisses Maß an Konsistenz erforderlich ist. Ein Beispiel hierfür sind kollaborative Online-Editoren, bei denen Benutzer möglicherweise verschiedene Versionen eines Dokuments sehen, aber das System im Laufe der Zeit die Konsistenz verbessert, indem es die Änderungen zwischen den Benutzern synchronisiert.

Hier ist ein vereinfachtes Fallbeispiel zur Veranschaulichung des Einsatzes von Metriken.

Angenommen, Sie betreiben eine Online-Nachrichtenplattform, auf der Redakteure Artikel veröffentlichen und Benutzer Kommentare hinterlassen können. In diesem Szenario gibt es mehrere wichtige Konsistenzanforderungen:
\begin{itemize}
\item Redakteure sollten in der Lage sein, Änderungen an Artikeln in Echtzeit vorzunehmen, ohne dass es zu langen Verzögerungen kommt.
\item Benutzer sollten die neuesten Artikel und Kommentare sehen können, wobei jedoch eine gewisse Verzögerung toleriert werden kann.
\end{itemize}
Um das Modell in dieser Anwendung zu implementieren, legen wir folgende Metriken und Schwellenwerte fest:
\begin{itemize}
\item Inkonsistenzdauer (ID): Die maximale Zeitdauer, während der ein Knoten im System eine inkonsistente Version der Daten haben darf. Zum Beispiel 5 Minuten für Artikel und 10 Minuten für Kommentare.
\item Inkonsistenzgröße (IS): Die maximale Anzahl von Inkonsistenzen, die in einem bestimmten Zeitraum auftreten dürfen. Zum Beispiel 10 Inkonsistenzen pro Stunde für Artikel und 20 Inkonsistenzen pro Stunde für Kommentare.
\end{itemize}

Die Online-Nachrichtenplattform verwendet in unserem Beispiel eine verteilte Datenbank, um Artikel und Kommentare zu speichern. Um das Konsistenzanforderungen umzusetzen, können Sie folgende Schritte ausführen:
\begin{itemize}
\item Bei jeder Aktualisierung der Daten (z. B. Hinzufügen/Ändern eines Artikels oder Hinzufügen eines Kommentars) messen Sie die Inkonsistenzdauer und die Inkonsistenzgröße. Diese Informationen können beispielsweise in Metadaten der Datenbank gespeichert werden.
\item Anhand der gemessenen ID und IS können Sie entscheiden, wann und wie oft Aktualisierungen zwischen den Knoten der verteilten Datenbank synchronisiert werden sollen. Wenn beispielsweise die ID oder IS für Artikel die festgelegten Schwellenwerte überschreitet, kann die Synchronisationsfrequenz erhöht werden, um die Konsistenz schneller wiederherzustellen.
\end{itemize}

Im Falle der Nichteinhaltung der festgelegten Metriken im CONIT-Modell (Inkonsistenzdauer oder Inkonsistenzgröße) kann das verteilte System verschiedene Maßnahmen ergreifen, um die Konsistenz zu verbessern oder die Auswirkungen der Inkonsistenzen zu minimieren. Diese Maßnahmen hängen von den spezifischen Anforderungen und Prioritäten des Systems ab.

Einige mögliche Maßnahmen bei Nichteinhaltung der Metriken sind:
\begin{itemize}
\item Erhöhen der Synchronisationsfrequenz: Das System kann die Frequenz erhöhen, mit der Updates zwischen den Knoten repliziert werden, um die Konsistenz schneller wiederherzustellen. Dies kann die Inkonsistenzdauer reduzieren, aber möglicherweise die Netzwerklast und Latenz erhöhen.
\item Benachrichtigung von Administratoren oder Entwicklern: Im Falle einer Nichteinhaltung der Metriken kann das System Benachrichtigungen an die verantwortlichen Administratoren oder Entwickler senden, damit sie die Situation untersuchen und angemessene Maßnahmen ergreifen können.
\item Anpassung der Schwellenwerte: Wenn die Nichteinhaltung der Metriken auf eine unzureichende Festlegung der Schwellenwerte zurückzuführen ist, können diese angepasst werden, um realistischere Ziele für das System zu setzen. Eine sorgfältige Analyse der Systemleistung und der Anwendungsanforderungen kann dazu beitragen, geeignete Schwellenwerte festzulegen.
\item Dynamische Anpassung der Konsistenzanforderungen: In einigen Fällen kann das System die Konsistenzanforderungen basierend auf der aktuellen Last oder den Prioritäten der Anwendung dynamisch anpassen. Zum Beispiel kann das System während Zeiten hoher Last eine lockere Konsistenz akzeptieren, um die Verfügbarkeit und Leistung aufrechtzuerhalten, und während Zeiten niedriger Last strengere Konsistenzanforderungen durchsetzen.
\end{itemize}
Es ist wichtig zu betonen, dass die Reaktion auf Nichteinhaltung der Metriken von den spezifischen Anforderungen und dem Kontext der Anwendung abhängt. Entwickler sollten sorgfältig abwägen, welche Maßnahmen am besten geeignet sind, um die Konsistenz in ihrem verteilten System aufrechtzuerhalten oder wiederherzustellen, bishin zur Einstellung des Dienstes. 




\subsubsection{Fork Consistency}

Fork Consistency ist ein neueres Konsistenzmodell, das im Bereich der verteilten Systeme verwendet wird. Bevor wir auf das Modell selbst eingehen, ist es wichtig, den Kontext von Konsistenzmodellen in verteilten Systemen zu verstehen. In verteilten Systemen arbeiten mehrere Knoten oder Server zusammen, um gemeinsam Daten und Rechenressourcen bereitzustellen. Eine der größten Herausforderungen dabei ist die Koordination und Synchronisation der Daten zwischen den Knoten, um eine konsistente Ansicht für die Clients sicherzustellen.
Traditionelle Konsistenzmodelle, wie Sequentielle Konsistenz und Linearisierbare Konsistenz, stellen strenge Anforderungen an die Reihenfolge, in der Operationen und ihre Ergebnisse zwischen den Knoten sichtbar sind. Diese Modelle sind zwar einfach zu verstehen und ermöglichen eine leichtere Programmierung, sie führen jedoch häufig zu Leistungseinbußen, da sie eine hohe Kommunikations- und Synchronisationslast erfordern.
In den letzten Jahren wurden daher neue, schwächere Konsistenzmodelle entwickelt, die bessere Leistung und Skalierbarkeit ermöglichen. Eines dieser Modelle ist die Fork-Konsistenz. Fork Consistency ist ein Konsistenzmodell, das die Sicherheitsanforderungen für Systeme mit mehreren Autoren lockert, indem es zulässt, dass Benutzer eine inkonsistente Ansicht der Daten sehen, solange sie auf getrennten \enquote{Forks} oder \enquote{Verzweigungen} des Systems arbeiten.
Die Hauptidee hinter Fork Consistency ist, dass das System nicht versucht, alle Änderungen sofort zwischen allen Knoten zu synchronisieren. Stattdessen werden Änderungen lokal an den Knoten vorgenommen und erst später, wenn nötig, mit anderen Knoten synchronisiert. Dabei entstehen \enquote{Forks}, also Verzweigungen in der Versionshistorie der Daten. Solange diese Verzweigungen getrennt bleiben, kann das System weiterarbeiten, ohne sich um die Konsistenz der Daten zwischen den Knoten sorgen zu müssen.
Während Fork Consistency einige Vorteile in Bezug auf Leistung und Skalierbarkeit bietet, hat es auch einige Nachteile und Herausforderungen:
\\\\
Ein Herausforderung ist, dass das Modell keine unmittelbare Synchronisation erfordert, daher können Benutzer eine inkonsistente Ansicht der Daten erhalten, was zu Anwendungsfehlern führen kann.
\\\\
Weiter müssen irgendwann die verschiedenen Verzweigungen der Daten wieder zusammengeführt werden. Dies kann ein komplexer Prozess sein, insbesondere wenn mehrere Autoren beteiligt sind und Konflikte in den Daten auftreten.
\\\\
Nicht zu letzt ist es eine schwierige Programmierung. Im Gegensatz zu strengeren Konsistenzmodellen müssen Entwickler, die mit Fork Consistency arbeiten, möglicherweise zusätzliche Anstrengungen unternehmen, um sicherzustellen, dass ihre Anwendungen korrekt funktionieren, wenn inkonsistente Daten vorhanden sind.
\\\\
Trotz dieser Herausforderungen bietet Fork Consistency in bestimmten Szenarien, insbesondere solchen mit einer hohen Anzahl von Knoten und Schreiboperationen, erhebliche Vorteile in Bezug auf Leistung und Skalierbarkeit. Um die Herausforderungen der Fork Consistency zu bewältigen, können Entwickler und Systemarchitekten verschiedene Strategien anwenden:
Konfliktlösung: Die Implementierung einer effizienten und robusten Strategie zur Konfliktlösung ist entscheidend, um die verschiedenen Verzweigungen der Daten erfolgreich zusammenzuführen. Dies kann beispielsweise durch den Einsatz von speziellen Algorithmen zur Zusammenführung von Daten, sogenannten Merge- oder Reconciliation-Algorithmen, erreicht werden. In einigen Fällen kann die Konfliktlösung auch manuell oder semi-manuell durch Benutzer oder Administratoren durchgeführt werden.
Anwendungsspezifische Konsistenz: In
\\\\
vielen Fällen ist es möglich, die Anforderungen an die Konsistenz auf Anwendungsebene zu definieren und anzupassen. Entwickler können dabei spezifische Mechanismen zur Handhabung von Inkonsistenzen implementieren, die den Anforderungen ihrer Anwendung am besten entsprechen. Ein Beispiel dafür ist die Verwendung von Convergent Replicated Data Types (CRDTs) oder Conflict-Free Replicated Data Types, die eine automatische Konfliktlösung ohne Koordination zwischen den Knoten ermöglichen.
\\\\
Um die Auswirkungen von Inkonsistenzen auf Benutzer und Anwendungen zu minimieren, können Systeme Strategien zur Steuerung der Sichtbarkeit von Änderungen auf Client-Seite implementieren. Beispielsweise können Clients dazu angehalten werden, nur konsistente Daten anzuzeigen, bis alle Verzweigungen erfolgreich zusammengeführt wurden. In einigen Fällen kann es sinnvoll sein, den Konsistenzgrad dynamisch an die aktuellen Anforderungen und Bedingungen des Systems anzupassen. Dies kann durch die Implementierung von Mechanismen erreicht werden, die es ermöglichen, zwischen verschiedenen Konsistenzmodellen zu wechseln, je nach Last, Netzwerkbedingungen oder Benutzeranforderungen.
Insgesamt bietet Fork Consistency ein interessantes und flexibles Konsistenzmodell für verteilte Systeme mit hoher Schreiblast und Skalierungsanforderungen. Um die damit verbundenen Herausforderungen zu bewältigen, ist es jedoch wichtig, sowohl auf System- als auch auf Anwendungsebene geeignete Mechanismen zur Handhabung von Inkonsistenzen, Konfliktlösung und Synchronisation zu implementieren.


\subsubsection{Multidimensionale Konsistenz }

Multidimensionale Konsistenz (Multidimensional Consistency) ist ein Konzept, das in verschiedenen Disziplinen wie Statistik, maschinellem Lernen, Datenanalyse und anderen verwandten Bereichen verwendet wird. Es bezieht sich auf das Maß der Übereinstimmung oder Konsistenz in den Daten oder Ergebnissen über verschiedene Dimensionen oder Aspekte eines Problems oder einer Analyse. Im Wesentlichen bedeutet es, dass die Ergebnisse in verschiedenen Dimensionen konsistent und plausibel sein sollten.

Die multidimensionale Konsistenz kann in verschiedenen Kontexten betrachtet werden:
\begin{itemize}
\item Datenqualität: Bei der Sammlung, Verarbeitung und Analyse von Daten ist es wichtig, die Qualität der Daten über verschiedene Dimensionen hinweg zu gewährleisten. Multidimensionale Konsistenz bedeutet hier, dass die Daten über verschiedene Attribute oder Dimensionen hinweg konsistent sind und keine Inkonsistenzen oder Widersprüche aufweisen, die die Analyse beeinträchtigen könnten.

\item Modellierung: In der Modellierung, insbesondere bei der Anwendung von maschinellem Lernen und statistischen Methoden, ist es wichtig, dass die Modelle, die auf den Daten trainiert werden, multidimensionale Konsistenz aufweisen. Dies bedeutet, dass die Modelle in der Lage sein sollten, konsistente und plausible Ergebnisse über verschiedene Dimensionen oder Aspekte des Problems hinweg zu liefern.

\item Ergebnisse und Interpretation: Die multidimensionale Konsistenz erstreckt sich auch auf die Ergebnisse und Interpretationen, die aus den Daten und Modellen abgeleitet werden. Es ist wichtig, dass die Ergebnisse über verschiedene Dimensionen oder Aspekte des Problems hinweg konsistent sind und plausibel erscheinen, um die Gültigkeit und Relevanz der Analyse zu gewährleisten.
\end{itemize}
Um die multidimensionale Konsistenz zu erreichen, können verschiedene Techniken und Ansätze verwendet werden, wie zum Beispiel:
\begin{itemize}
\item Datenvorverarbeitung: Sorgfältige Datenaufbereitung, -bereinigung und -validierung sind entscheidend, um Inkonsistenzen und Widersprüche in den Daten zu identifizieren und zu beheben.
\item Auswahl von Merkmalen und Dimensionen: Eine sorgfältige Auswahl von Merkmalen und Dimensionen kann dazu beitragen, die Relevanz und Konsistenz der Daten und Modelle zu gewährleisten.
\item Kreuzvalidierung: Bei der Anwendung von maschinellem Lernen und statistischen Methoden kann die Kreuzvalidierung dazu beitragen, die Konsistenz der Modelle und Ergebnisse über verschiedene Datensätze und Dimensionen hinweg zu überprüfen.
\item Visualisierung und Exploration: Die Visualisierung und Exploration der Daten und Ergebnisse kann dazu beitragen, Muster und Zusammenhänge aufzudecken, die zur multidimensionalen Konsistenz beitragen.
\end{itemize}
Insgesamt ist die multidimensionale Konsistenz ein wichtiger Aspekt bei der Gewährleistung der Qualität und Gültigkeit von Daten, Modellen und Ergebnissen in verschiedenen Anwendungsbereichen. Es trägt dazu bei, eine konsistente und fundierte Grundlage für Entscheidungen und Maßnahmen auf der Grundlage von Datenanalysen zu schaffen.
\\\\
Ein einfaches Beispiel für multidimensionale Konsistenz könnte die Analyse von Verkaufsdaten in einem Einzelhandelsgeschäft sein. Angenommen, wir haben folgende Verkaufsdaten für verschiedene Produkte (A, B und C) und Regionen (Nord, Süd, Ost und West):

\begin{table}[ht!]
\centering
\begin{tabular}{@{}llr@{}}
Produkt & Region & Verkaufszahlen \\ 
A       & Nord   & 100            \\
A       & Süd    & 90             \\
A       & Ost    & 110            \\
A       & West   & 120            \\
B       & Nord   & 150            \\
B       & Süd    & 140            \\
B       & Ost    & 160            \\
B       & West   & 170            \\
C       & Nord   & 50             \\
C       & Süd    & 45             \\
C       & Ost    & 55             \\
C       & West   & 60             \\ 
\end{tabular}
\caption{Verkaufsdaten für verschiedene Produkte und Regionen}
\end{table}
Die multidimensionale Konsistenz in diesem Beispiel bezieht sich auf die Konsistenz der Verkaufszahlen über die verschiedenen Produkte und Regionen hinweg. Um die multidimensionale Konsistenz zu überprüfen, sollten wir die Verkaufszahlen sowohl in Bezug auf die Produkte als auch auf die Regionen analysieren.

Wir können zunächst die Verkaufszahlen für jedes Produkt über alle Regionen hinweg überprüfen:
\begin{itemize}
    \item Produkt A: 100 (Nord) + 90 (Süd) + 110 (Ost) + 120 (West) = 420
    \item Produkt B: 150 (Nord) + 140 (Süd) + 160 (Ost) + 170 (West) = 620
    \item Produkt C: 50 (Nord) + 45 (Süd) + 55 (Ost) + 60 (West) = 210
\end{itemize}
Dann können wir die Verkaufszahlen für jede Region über alle Produkte hinweg überprüfen:
\begin{itemize}
    \item Region Nord: 100 (A) + 150 (B) + 50 (C) = 300
    \item Region Süd: 90 (A) + 140 (B) + 45 (C) = 275
    \item Region Ost: 110 (A) + 160 (B) + 55 (C) = 325
    \item Region West: 120 (A) + 170 (B) + 60 (C) = 350
\end{itemize}

In diesem Beispiel scheinen die Verkaufszahlen über die verschiedenen Produkte und Regionen hinweg konsistent zu sein. Es gibt keine offensichtlichen Inkonsistenzen oder Widersprüche in den Daten, und die Zahlen erscheinen plausibel. 

\subsubsection{Adaptable Consistency}

Adaptable Consistency ist ein Konzept, das in verteilten Systemen verwendet wird, um die Konsistenzanforderungen an die spezifischen Bedürfnisse und Anforderungen einer Anwendung oder Situation anzupassen. Es ermöglicht, dass die Konsistenz in einem verteilten System flexibel gestaltet wird, um sowohl Leistung als auch Datenkonsistenz zu optimieren, je nach den Anforderungen der Anwendung und den zugrunde liegenden Systembedingungen.

Ein  Beispiel für adaptable Consistency könnte ein verteiltes Key-Value-Store-System sein, das sowohl eventual consistency als auch strong consistency unterstützt, abhängig von den Anforderungen der Anwendung oder der Situation.

Angenommen, wir haben ein verteiltes System mit drei Knoten (A, B und C), die einen Key-Value-Store verwalten. Für dieses Beispiel verwenden wir einfache Lese- und Schreiboperationen.
\begin{itemize}
\item In einer Situation, in der die Latenz der Operationen von entscheidender Bedeutung ist und eine gewisse Inkonsistenz toleriert werden kann, könnte das System eventual consistency verwenden. In diesem Fall schreibt der Client die Daten an einen der Knoten (z.B. Knoten A) und das Update wird später auf die anderen Knoten (B und C) repliziert. Bei Leseanfragen können die Clients die Daten von jedem Knoten lesen, auch wenn sie möglicherweise noch nicht konsistent sind, was zu einer niedrigeren Latenz führt.
\item In einer anderen Situation, in der Datenkonsistenz von größter Bedeutung ist und höhere Latenzen toleriert werden können, könnte das System strong consistency verwenden. In diesem Fall schreibt der Client die Daten an einen der Knoten (z.B. Knoten A) und wartet, bis das Update auf die anderen Knoten (B und C) repliziert wurde, bevor die Schreiboperation als erfolgreich gemeldet wird. Bei Leseanfragen erhalten die Clients konsistente Daten, da das System sicherstellt, dass alle Knoten den gleichen Wert für einen bestimmten Schlüssel haben.
\end{itemize}
Das verteilte System kann seine Konsistenzstrategie an die Anforderungen der Anwendung oder die Systembedingungen anpassen, z.B. indem es eine niedrigere Konsistenzstufe wählt, wenn die Netzwerklatenz hoch ist oder ein Knotenausfall auftritt, oder indem es eine höhere Konsistenzstufe wählt, wenn die Datenintegrität von entscheidender Bedeutung ist. Durch die Anpassung der Konsistenzstrategie kann das System eine bessere Balance zwischen Leistung und Datenkonsistenz erreichen und so den spezifischen Bedürfnissen und Anforderungen der Anwendung gerecht werden.

\subsubsection{View Consistency}

View Consistency ist ein Konsistenzmodell, das in verteilten Systemen verwendet wird, um sicherzustellen, dass alle Prozesse oder Knoten, die auf Daten zugreifen, eine konsistente Sicht auf diese Daten haben. Im Wesentlichen bedeutet dies, dass alle Prozesse, die zur gleichen Zeit auf Daten zugreifen, dieselben Werte sehen. View Consistency ist ein schwächeres Konsistenzmodell als Strong Consistency, aber stärker als Eventual Consistency.
\\\\
Ein  Beispiel für View Consistency ist ein verteiltes Chat-System, in dem Nachrichten zwischen Benutzern ausgetauscht werden.

Angenommen, wir haben ein verteiltes Chat-System mit drei Knoten (A, B und C), die Nachrichten für verschiedene Benutzer speichern und verwalten. In diesem System gibt es zwei Benutzer, Alice und Bob, die miteinander chatten.
\begin{itemize}
\item Alice sendet eine Nachricht \enquote{Hallo} an Bob. Die Nachricht wird zuerst auf Knoten A gespeichert.
\item Das System repliziert die Nachricht auf den anderen Knoten (B und C) gemäß der View Consistency-Anforderung.
\item Während der Replikation können sowohl Alice als auch Bob weiterhin Nachrichten senden und empfangen. Das System stellt jedoch sicher, dass alle Nachrichten, die während der Replikation gesendet werden, in der richtigen Reihenfolge auf allen Knoten gespeichert werden, um die View Consistency aufrechtzuerhalten.
\item Sobald die Replikation abgeschlossen ist, sehen sowohl Alice als auch Bob die gleiche Sicht auf die Nachrichten, einschließlich der Reihenfolge, in der sie gesendet wurden.
\end{itemize}
In diesem Beispiel ermöglicht View Consistency, dass alle Benutzer eine konsistente Sicht auf die Chat-Nachrichten haben, ohne die strengeren Anforderungen von Strong Consistency, die höhere Latenzen verursachen könnten. Gleichzeitig gewährleistet es eine höhere Konsistenz als Eventual Consistency, bei der Inkonsistenzen zwischen Knoten für einen gewissen Zeitraum toleriert werden.

\subsubsection{Write Write Conflicts}
In verteilten Systemen können Write-Write-Konflikte auftreten, wenn zwei oder mehr Schreibvorgänge von verschiedenen Clients gleichzeitig oder nahezu gleichzeitig auf denselben Datensatz angewendet werden. Solche Konflikte sind von Bedeutung, da sie zu Inkonsistenzen führen können, wenn die Schreibvorgänge nicht ordnungsgemäß koordiniert und synchronisiert werden. Write-Write-Konflikte stellen eine Herausforderung für verteilte Systeme dar, da sie die Integrität und Konsistenz der Daten beeinträchtigen können, wenn sie nicht angemessen behandelt werden.

Ein Beispiel für einen Write-Write-Konflikt könnte in einem verteilten Kollaborationswerkzeug wie einem Textdokument auftreten, an dem mehrere Benutzer gleichzeitig arbeiten. Wenn zwei Benutzer zur gleichen Zeit unterschiedliche Änderungen am gleichen Absatz vornehmen, entsteht ein Write-Write-Konflikt, da das System entscheiden muss, welche der beiden Änderungen Vorrang haben sollte oder wie diese Änderungen zusammengeführt werden können, ohne den Inhalt des Dokuments zu beeinträchtigen.

Um Write-Write-Konflikte in verteilten Systemen zu bewältigen, können verschiedene Ansätze verfolgt werden, darunter:
\begin{itemize}
\item Sperren und Transaktionen: Das System kann exklusive Sperren für Datensätze verwenden, um sicherzustellen, dass nur ein Client gleichzeitig auf einen Datensatz zugreifen kann. Transaktionen können ebenfalls eingesetzt werden, um mehrere Schreibvorgänge atomar zu behandeln und sicherzustellen, dass sie entweder vollständig abgeschlossen oder abgebrochen werden, um Konsistenzprobleme zu vermeiden.
Datenbanksysteme: Relationale Datenbanksysteme wie PostgreSQL und MySQL verwenden Sperren und Transaktionen, um Konsistenz und Datenintegrität in verteilten Umgebungen sicherzustellen. Bei diesen Systemen können Clients Transaktionen verwenden, um mehrere Schreibvorgänge zusammenzufassen und sicherzustellen, dass sie entweder vollständig abgeschlossen oder abgebrochen werden, um Konsistenzprobleme zu vermeiden.
\item Optimistische Nebenläufigkeitskontrolle: Bei dieser Methode wird angenommen, dass Konflikte selten auftreten. Clients können ihre Schreibvorgänge ausführen, ohne auf Sperren zu warten. Vor dem Commit wird überprüft, ob Konflikte aufgetreten sind. Im Falle eines Konflikts wird eine der Transaktionen zurückgerollt und später erneut ausgeführt.
Apache Cassandra: Dieses verteilte NoSQL-Datenbanksystem verwendet die optimistische Nebenläufigkeitskontrolle, um hohe Leistung und Skalierbarkeit zu ermöglichen. Clients führen Schreibvorgänge aus, ohne auf Sperren zu warten, und das System verwendet Zeitstempel, um Konflikte aufzulösen, wenn sie auftreten. Die neueste Änderung (basierend auf dem Zeitstempel) hat Vorrang und überschreibt ältere Änderungen.
\item Versionskontrolle und Zusammenführungsstrategien: In einigen Systemen, insbesondere in kollaborativen Anwendungen, kann es sinnvoll sein, verschiedene Versionen der Daten beizubehalten und Strategien zum Zusammenführen von Änderungen zu implementieren, die gleichzeitig vorgenommen wurden. Diese Strategien können automatisch sein oder Benutzerinteraktion erfordern, um die Zusammenführung abzuschließen.
Ein anderes Beispiel neben Google Docs ist Git. Git ist ein verteiltes Versionskontrollsystem, das häufig von Softwareentwicklern zur Verwaltung von Quellcode verwendet wird. Git behält verschiedene Versionen der Dateien bei und ermöglicht es Benutzern, Änderungen zusammenzuführen, die von verschiedenen Entwicklern gleichzeitig vorgenommen wurden. Wenn ein Write-Write-Konflikt auftritt, fordert Git den Benutzer auf, die Konflikte manuell zu lösen, bevor die Änderungen zusammengeführt werden können.
\end{itemize}
Die aller Beste Strategie ist allerdings dem grundlegenden Problem aus dem Wege zu gehen und die Schreibvorgänge auf einem System zu sequenzialisieren. Ein gängiger Ansatz zur Sequenzialisierung von Schreibvorgängen in verteilten Systemen besteht darin, einen sogenannten \enquote{Primary} oder \enquote{Master} Knoten zu verwenden. Bei diesem Ansatz werden alle Schreibvorgänge an einen zentralen Knoten gesendet, der dafür verantwortlich ist, die Schreibvorgänge in einer bestimmten Reihenfolge auszuführen und die aktualisierten Daten an die anderen Knoten im verteilten System zu replizieren.

Ein Beispiel für ein verteiltes System, das Schreibzugriffe sequenziell verarbeitet, ist MongoDB im Primary-Secondary-Modus. In dieser Konfiguration empfängt der Primary-Knoten alle Schreibvorgänge und führt sie in einer bestimmten Reihenfolge aus, bevor die aktualisierten Daten an die sekundären Knoten repliziert werden. Die sekundären Knoten können Lesevorgänge ausführen, um die Last auf dem Primary-Knoten zu verringern.
\\\\
Es gibt jedoch auch Nachteile bei der Verwendung von sequenziellen Schreibvorgängen in verteilten Systemen. Durch die Sequenzialisierung der Schreibvorgänge kann die Skalierbarkeit eingeschränkt werden, da der zentrale Knoten ein Engpass für die Leistung und Verfügbarkeit des Systems werden kann. Aus diesem Grund verwenden viele verteilte Systeme die anderen Ansätze trotz ihrer Nachteile. Die Synchronization der Master mit den Replikationen ist zudem auch wiederum eine eigene Herausforderung die aber nicht nur in diesem sondern jedem anderen Szenario besteht.  

Jede dieser Lösungsstrategien hat ihre eigenen Vor- und Nachteile, und die Wahl der am besten geeigneten Strategie hängt von den spezifischen Anforderungen und Merkmalen des verteilten Systems ab. Die Beispiele hier veranschaulichen die Vielfalt der Ansätze, die in der Praxis angewendet werden, um Write-Write-Konflikte zu bewältigen und die Konsistenz und Datenintegrität in verteilten Systemen aufrechtzuerhalten. Dennoch werden in der Praxis auch immer wieder Strategien wie replicated-write protocols eingesetzt die ein hohes Mass an Abstimmung und Konsenz bedürfen und in einem späteren Kapitel im Kontext von Raft und Paxos nochmal betrachtet werden.  



	



