\documentclass{article}

\begin{document}

\title{Übungsaufgabe: Application-Stub, Client-Stub und Skeleton}
\maketitle

\section*{Aufgabe}

Die Konzepte des Application-Stubs, Client-Stubs und Skeletons sind zentrale Elemente in der Architektur von verteilten Systemen. In dieser Übung sollen Sie Ihr Verständnis dieser Konzepte durch die Beantwortung der folgenden Fragen vertiefen.

\begin{enumerate}
    \item Definieren Sie, was ein Application-Stub, ein Client-Stub und ein Skeleton in einem verteilten System sind.

    \item Beschreiben Sie, wie diese drei Konzepte zusammenwirken, um die Kommunikation in einem verteilten System zu ermöglichen. Nutzen Sie ein praktisches Beispiel, um Ihre Beschreibung zu verdeutlichen. Nehmen Sie an, Sie hätten eine Anwendung, die einen entfernten Web-Service zur Wettervorhersage nutzt. 

    \item Was sind die Aufgaben des Client-Stubs und des Skeletons im Kontext dieser Wettervorhersage-Anwendung?

    \item Stellen Sie sich vor, es tritt ein Fehler bei der Kommunikation mit dem Web-Service auf. Wo würden Sie bei der Fehlersuche ansetzen, und welche Rolle spielen der Application-Stub, der Client-Stub und das Skeleton dabei?

\end{enumerate}

\end{document}
