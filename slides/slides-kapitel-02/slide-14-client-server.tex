%\section{Einleitung}
\subsection{Client-Server-Architektur}
\begin{frame}
  \frametitle{Client-Server-Architektur}
  \framesubtitle{Modell}
  \begin{itemize}
    \item In dieser Architektur gibt es zwei Hauptkomponenten
    \item Client - nutzt (Consumer)
    \item Server - bietet an (Provider)
    \item Ein spezielle \enquote{2-Tier Architektur}
    \item Front-End-Client und einem Back-End-Server
  \end{itemize}
\end{frame}

\begin{frame}
  \frametitle{Client-Server-Architektur}
  \framesubtitle{Beispiele}
  \begin{itemize}
    \item Webanwendungen
    \item Datenbankanwendungen
  \end{itemize}
\end{frame}

\begin{frame}
  \frametitle{Client-Server-Architektur}
  \framesubtitle{Kaskadenartige Kommunikation}
  \begin{itemize}
    \item n-Tier-Architektur sind Tiers sowohl Consumer- als auch Provider
    \item Auswirkung auf Latenz, Komplexität, Fehleranfälligkeit, Skalierbarkeit
    \item Einführung von Optimierungsmechanismen, wie Caching, Lastverteilung, asynchrone Kommunikation 
  \end{itemize}
\end{frame}

\begin{frame}
  \frametitle{Client-Server-Architektur}
  \framesubtitle{SSP - Stub/Skeleton Chains}
  \begin{itemize}
    \item Eine Implementierungsform von Interface-Definitionen zur Kommunikation
    \item Methodenaufrufe von einer Schicht zur anderen
    \item Stub ist ein Platzhalter für eine entfernte Methode (Lokal oder Remote)
    \item SSP Stub/Skeleton Chains sind eine Sequenz von Stubs und Skeletons
    \item Gemeinsame Sprache zwischen verschiedenen Schichten oder Tiers
  \end{itemize}
\end{frame}