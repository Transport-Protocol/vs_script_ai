%\section{Einleitung}
\subsection{Middleware Architektur}
\begin{frame}
  \frametitle{Middleware Architektur}
  \framesubtitle{Idee}
  \begin{itemize}
    \item Kapselung der Aufgabe
    \item Eigene Schicht für die Herausforderung der verteilten Systemen 
    \item Schicht wird als \enquote{Middleware-}Schicht bezeichnet
    \item Bietet im besten Fall die Schnittstelle des kohärenten Systems
  \end{itemize}
\end{frame}

\begin{frame}
  \frametitle{Middleware Architektur}
  \framesubtitle{Definition nach Tanenbaum}
  \begin{itemize}
    \item Bereitstellung eines breiten Angebotes für die Kommunikation
    \item Dem (Un-)Marshaling Prozess 
    \item Protokolle zur Namensauflösung
    \item Sicherheitsprotokolle
    \item Mechanismen zur Steigerung der Skalierung
  \end{itemize}
\end{frame}

\begin{frame}
  \frametitle{Middleware Architektur}
  \framesubtitle{Erweiterte Anforderungen}
  \begin{itemize}
    \item Mehrere Sprachen
    \item Mehreren Betriebssystemen und Hardwaretypen
    \item Mehrere Netzwerkprotokolle zur Einbindung von Internetstrukturen
  \end{itemize}
\end{frame}

\begin{frame}
  \frametitle{Middleware Architektur}
  \framesubtitle{Kommunikation}
  \begin{itemize}
    \item Point to point
    \item Point to multipoint
    \item Publish/subscribe
    \item Client/Server, Request/Reply
    \item Mobile code
    \item Virtual shared
  \end{itemize}
\end{frame}

\begin{frame}
  \frametitle{Middleware Architektur}
  \framesubtitle{Message Oriented Middleware (MOM)}
  \begin{itemize}
      \item Direkte Kommunikation zwischen Anwendungen (Message Passing)
      \item Indirekte Kommunikation über eine Warteschlange (Message Queueing)
      \item Herausgeber stellt dem Abonnenten Nachrichten zur Verfügung (Publish \& Subscribe)
  \end{itemize}
\end{frame}

