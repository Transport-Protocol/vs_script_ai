%\section{Einleitung}
\subsection{Virtualisierung}
\begin{frame}
  \frametitle{Virtualisierung}
  \framesubtitle{Motivation}
  \begin{itemize}
    \item Isolation und Sicherheit 
    \item Optimaler Resourcen-Nutzung
    \item Flexibilität und Skalierbarkeit
    \item Logische Einheiten (VM)
    \item Einfach erstellt, gelöscht, migriert oder skaliert
    \item Schnelles Hinzufügen oder Entfernen von Ressourcen
    \item Einfache Verwaltung, Wartung und Testbarkeit
  \end{itemize}
\end{frame}


\begin{frame}
  \frametitle{Virtualisierung}
  \framesubtitle{Geschichte}
  \begin{itemize}
    \item Wurzel in den frühen Jahren der modernen Informatik (1960)
    \item Erste IBM Virtualisierungsplattform: CP-40-System
    \item Time-Sharing und Multi-User-Betriebssysteme
    \item Einführung von Netzwerken in den 1980 ermöglichte neue Verteilung
    \item 1990 WWW enabler für heutige Struktur und Kommunikationstechnologien
    \item Virtualisierung in den 2000er durch VMware und Xen
    \item Containertechnologien wie Docker in den 2010er
  \end{itemize}
\end{frame}


\begin{frame}
  \frametitle{Virtualisierung}
  \framesubtitle{Fokus}
  \begin{itemize}
    \item Hardware-Virtualisierung
    \item Betriebssystem-Virtualisierung
    \item Anwendungs-Virtualisierung
    \item Speicher- und Netzwerkvirtualisierung
  \end{itemize}
\end{frame}

\begin{frame}
  \frametitle{Virtualisierung}
  \framesubtitle{Hardware-Virtualisierung }
  \begin{itemize}
    \item Hardware in virtuelle Instanzen aufgeteilt
    \item Effiziente Nutzung von Ressourcen und die Isolierung
    \item Verschiedener Systeme und Anwendungen auf der gleichen physischen Hardware
    \item Nutzt Funktionen der CPU und anderer Komponenten für HW-Unterstützung
    \item Zentrale Herausforderung Leistung
  \end{itemize}
\end{frame}


\begin{frame}
  \frametitle{Virtualisierung}
  \framesubtitle{Hardware-Virtualisierung Herausforderungen}
  \begin{itemize}
    \item Teilung der Resourcen und Isolation
    \item Management von Input/Output (I/O)-Operationen\\Technologien wie I/O-Virtualisierung und Direct Memory Access (DMA)-Remapping sollen Effekte mindern
    \item Kompatibilität\\
    Besonderer Fokus auf Hypervisor
  \end{itemize}
\end{frame}


\begin{frame}
  \frametitle{Virtualisierung}
  \framesubtitle{Hypervisor}
  \begin{itemize}
    \item Typ 1-Hypervisoren (auch Bare-Metal-Hypervisoren)\\ 
    Typ 1-Hypervisoren sind VMware ESXi und Microsoft Hyper-V
    \item Typ 2-Hypervisoren \\
    VMware Workstation und Oracle VirtualBox
  \end{itemize}
\end{frame}

\begin{frame}
  \frametitle{Virtualisierung}
  \framesubtitle{Hardware-Virtualisierung Ansätze}
  \begin{itemize}
    \item Vollvirtualisierung 
    \item Paravirtualisierung
  \end{itemize}
\end{frame}

\begin{frame}
  \frametitle{Virtualisierung}
  \framesubtitle{Betriebssystemvirtualisierung}
  \begin{itemize}
    \item Logischen Einheiten Container 
    \item Gleicher Kernel und gleiche Systembibliotheken
    \item Basis ist gleiche Betriebssystem API
  \end{itemize}
\end{frame}

\begin{frame}
  \frametitle{Virtualisierung}
  \framesubtitle{Betriebssystemvirtualisierung Isolation}
  \begin{itemize}
    \item Herausforderung Isolierung der Container
    \item Trennung nicht so stark wie bei VM
    \item Ansätze durch AppArmor, SELinux und Seccomp 
  \end{itemize}
\end{frame}

\begin{frame}
  \frametitle{Virtualisierung}
  \framesubtitle{Vor- und Nachteile}
  \begin{itemize}
    \item V: Leistung
    \item V: Skalierbarkeit
    \item V: Kosteneffizienz
    \item N: Sicherheit
    \item N: Kompatibilität 
  \end{itemize}
\end{frame}

\begin{frame}
  \frametitle{Virtualisierung}
  \framesubtitle{Betriebssystemvirtualisierung Beispieltechnologien}
  \begin{itemize}
    \item Kubernetes
    \item Docker
    \item OpenStack
  \end{itemize}
\end{frame}

\begin{frame}
  \frametitle{Virtualisierung}
  \framesubtitle{Kubernetes}
  \begin{itemize}
    \item Open-Source-Orchestrierungssystem/ Cluster-Management-System
    \item Automatisierung der Bereitstellung, Skalierung und Verwaltung
    \item Ursprünglich von Google entwickelt 
    \item Kubernetes unterstützt z.B. Containerd und CRI-O (nicht  Docker-Daemons, Container Runtime Interface (CRI) unterstützt docker images)
  \end{itemize}
\end{frame}

\begin{frame}
  \frametitle{Virtualisierung}
  \framesubtitle{Docker}
  \begin{itemize}
    \item Open-Source-Plattform für die Containerisierung
    \item Der Ursprung vieler Entwicklungen
    \item Bietet eine leichtgewichtige Virtualisierung
    \item Docker-Container sind plattformübergreifend
    \item Der Begriff docker ist mehrdeutig (Docker-Daemon, Docker-CLI, Docker-Image)
  \end{itemize}
\end{frame}

\begin{frame}
  \frametitle{Virtualisierung}
  \framesubtitle{Docker Alternativen}
  \begin{itemize}
    \item Podman
    \item Buildah
    \item LXC (Linux Containers)
    \item rkt (ausgesprochen "Rocket" - obsolate) 
    \item containerd
    \item CRI-O
  \end{itemize}
\end{frame}

\begin{frame}
  \frametitle{Virtualisierung}
  \framesubtitle{OpenStack}
  \begin{itemize}
    \item Open-Source-Cloud-Computing-Plattform, die Infrastruktur als Service (IaaS) bietet
    \item Eigene Cloud-Infrastruktur mit verschiedenen Komponenten
    \item Kann in kubernetes eingesetzt werden
  \end{itemize}
\end{frame}

\begin{frame}
  \frametitle{Virtualisierung}
  \framesubtitle{Speichervirtualisierung}
  \begin{itemize}
    \item Physische Speicherressourcen in einem logischen Pool 
    \item Dynamisch und flexibel Nutzern zuweisbar
    \item Einfachere Verwaltung von Speicherressourcen 
  \end{itemize}
\end{frame}

\begin{frame}
  \frametitle{Virtualisierung}
  \framesubtitle{Speichervirtualisierung - SAN}
  \begin{itemize}
    \item Storage Area Networks (SANs)
    \item SAN ist ein dediziertes Hochgeschwindigkeitsnetzwerk
    \item Einfachere Verwaltung von Speicherressourcen 
  \end{itemize}
\end{frame}


\begin{frame}
  \frametitle{Virtualisierung}
  \framesubtitle{Netzwerkvirtualisierung}
  \begin{itemize}
    \item Physische Netzwerkressourcen in logische Einheiten abstrahiert
    \item Aufbau virtueller Netzwerke
    \item Beispiel für Netzwerkvirtualisierung ist SDN
  \end{itemize}
\end{frame}

\begin{frame}
  \frametitle{Virtualisierung}
  \framesubtitle{Desktopvirtualisierung}
  \begin{itemize}
    \item VDI-Lösungen wie VMware Horizon oder Windows 365
    \item Verbesserte Verwaltung und Wartung von Desktop-Betriebssystemen
    \item Desktops zentral verwaltet
    \item Bisher relativ hohe Kosten
  \end{itemize}
\end{frame}

\begin{frame}
  \frametitle{Virtualisierung}
  \framesubtitle{Virtualisierung als Dienst}
  \begin{itemize}
    \item Software as a Service (SaaS)
    \item Platform as a Service (PaaS)
    \item Infrastructure as a Service (IaaS)
  \end{itemize}
\end{frame}

\begin{frame}
  \frametitle{Virtualisierung}
  \framesubtitle{Virtualisierung als Dienst - Vorteile}
  \begin{itemize}
    \item Einfachere Bereitstellung und Skalierung
    \item Kosteneffizienz
    \item Fokus auf Kernkompetenzen
    \item Globale Präsenz und Leistung
    \item Erleichterte Integration und Zusammenarbeit
    \item Erhöhte Sicherheit und Compliance
  \end{itemize}
\end{frame}

\begin{frame}
  \frametitle{Virtualisierung}
  \framesubtitle{Virtualisierung als Dienst - Bietet}
  \begin{itemize}
    \item Investition
    \item Sicherheit und Datenschutz
    \item Optimierung der Anwendungsleistung
    \item Kundensupport und Service Level Agreements
  \end{itemize}
\end{frame}

\begin{frame}
  \frametitle{Virtualisierung}
  \framesubtitle{Virtualisierung als Dienst - Schwierigkeiten}
  \begin{itemize}
    \item Kosten
    \item Technische Kompatibilität
    \item Datenmigration
    \item Fehlende Regulierung
  \end{itemize}
\end{frame}

\begin{frame}
  \frametitle{Virtualisierung}
  \framesubtitle{Virtualisierung als Dienst - Beispiel Netflix Deployment}
  \begin{itemize}
    \item Frontend und API
    \item Microservices
    \item Container-Orchestrierung
    \item Datenbanken und Caching
    \item Datenspeicherung
    \item Content Delivery
    \item Big Data und Analyse
    \item Monitoring und Logging
    \item Sicherheit
    \item Automatisierung und Infrastruktur als Code
    \item CI/CD (Continuous Integration und Continuous Deployment)
    \item Resilienz und Fehlertoleranz
  \end{itemize}
\end{frame}

\begin{frame}
  \frametitle{Virtualisierung}
  \framesubtitle{Virtualisierung als Dienst - Beispiel Netflix Software}
  \begin{itemize}
    \item Netflix OSS (Open Source Software) ist eine Sammlung von Open-Source-Projekten
    \item Eureka: Ein Service-Discovery-System
    \item Hystrix: Eine Latenz- und Fehler-Toleranz-Bibliothek
    \item Chaos Monkey: Ein Tool zur Überprüfung der Fehlertoleranz 
    \item Dynomite: Eine hochverfügbare, verteilte und skalierbare Datenbank-Engine
  \end{itemize}
\end{frame}

 \begin{frame}
  \frametitle{Virtualisierung}
  \framesubtitle{Virtualisierung als Dienst - Beispiel Netflix Software}
  \begin{itemize}   
    \item Cassandra: Eine hochverfügbare, verteilte NoSQL-Datenbank
    \item Spinnaker: Eine Multi-Cloud-Continuous-Delivery-Plattform
    \item Atlas: Ein skalierbares und erweiterbares Monitoring-System
    \item Lemur: Ein Tool zur Verwaltung von TLS-Zertifikaten
    \item Titus: Netflix' hauseigener Container-Orchestrierungs-Service
    \item Genie: Eine Plattform für die Verwaltung und Ausführung von Big Data-Jobs
    \item Weiter Client-Bibliotheken und SDKs
  \end{itemize}
\end{frame}


