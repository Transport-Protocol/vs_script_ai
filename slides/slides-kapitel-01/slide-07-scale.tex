\section{Skalierung}
\subsection{Das Ziel}
\begin{frame}
  \frametitle{Skalierung}
  \framesubtitle{Das Ziel}
  \begin{itemize}
    \item Mit wachsenden Anforderungen und steigenden Benutzerzahlen umgehen
    \item Nicht Leistung oder Stabilität verlieren
    \item Funktionalität und Leistungsfähigkeit bei Bedarf anpassbar
  \end{itemize}
\end{frame}

\subsection{Dimensionen}
\begin{frame}
  \frametitle{Skalierung}
  \framesubtitle{Horizontal und Vertikal}
  \begin{itemize}
    \item Horizontal: Zusätzliche Knoten oder Server
    \item Vertikal: Mehr Ressourcen auf einem Knoten oder Server 
    \item Load-Balancer spielen eine besondere Rolle
  \end{itemize}
\end{frame}

\subsection{Unterziele}
\begin{frame}
  \frametitle{Skalierung}
  \framesubtitle{Unterziele}
  \begin{itemize}
    \item Räumliche Skalierung
    \item Verwaltungs- oder adminstrative- Skalierbarkeit
    \item Funktionale Skalierung
  \end{itemize}
\end{frame}

\subsection{Funktionale Skalierung}
\begin{frame}
  \frametitle{Skalierung}
  \framesubtitle{Funktionale Skalierung}
  \begin{itemize}
    \item Little's Law
    \item Amdahlsche's Law
    \item Gustafsons' Law 
  \end{itemize}
\end{frame}

\subsection{Speedup}
\begin{frame}
  \frametitle{Skalierung}
  \framesubtitle{Speedup}
  \begin{itemize}
    \item Idealer Speedup
    \item Realer Speedup
    \item Super-Speedup
  \end{itemize}
\end{frame}

\begin{frame}
  \frametitle{Skalierung}
  \framesubtitle{Bedeutung Parallelisierung am Beispiel von Sort}
  \begin{itemize}
    \item Quicksort
    \item Mergesort
  \end{itemize}
\end{frame}