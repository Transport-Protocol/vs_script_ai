\documentclass{beamer}

\usetheme{CambridgeUS}
\usecolortheme{dolphin}

\title{Verteilte Systeme}
\subtitle{Kommunikation, Koordination und Konsistenz}
\author{Prof. Dr. Martin Becke}
\date{\today}

\begin{document}

\begin{frame}
    \titlepage
\end{frame}

\begin{frame}{Kommunikation in verteilten Systemen}
    Kommunikation ist das Nervensystem verteilter Systeme. Sie ermöglicht den Informationsaustausch und die Koordination zwischen Knoten. Verschiedene Faktoren beeinflussen die Kommunikation, darunter Topologie, Eigenschaften, Protokolle und Serialisierungsformate. \newline
    Vergleichbar mit dem Straßennetz in einer Stadt: Es bestimmt, wie effizient Menschen und Güter von A nach B gelangen.
\end{frame}

\begin{frame}{Kommunikationstopologien}
    Die Topologie eines verteilten Systems beeinflusst Effizienz und Zuverlässigkeit. Gängige Topologien sind:
    \begin{itemize}
        \item \textbf{One-to-All:} Ein Knoten sendet an alle anderen. Vergleichbar mit einer Radioübertragung.
        \item \textbf{Spanning Tree:} Nachrichten werden entlang eines Baumes weitergeleitet. Vergleichbar mit einem Flusssystem.
        \item \textbf{Flooding:} Jeder Knoten sendet an alle Nachbarn. Vergleichbar mit einer Kettenreaktion.
        \item \textbf{Gossip:} Jeder Knoten sendet an einige zufällig ausgewählte Nachbarn. Vergleichbar mit der Verbreitung eines Gerüchts.
    \end{itemize}
\end{frame}

\begin{frame}{Kommunikationseigenschaften}
    Nachrichten können abhängig von den Anwendungsanforderungen folgende Eigenschaften haben:
    \begin{itemize}
        \item \textbf{Persistent:} Nachrichten werden dauerhaft gespeichert (z. B. ein Brief).
        \item \textbf{Transient:} Nachrichten werden nur vorübergehend gespeichert (z. B. ein Telefongespräch).
    \end{itemize}
\end{frame}

\begin{frame}{Kommunikationseigenschaften: Synchron vs. Asynchron}
    Die zeitliche Kopplung der Kommunikationspartner beeinflusst das Systemdesign:
    \begin{itemize}
        \item \textbf{Synchron:} Der Sender wartet auf eine Antwort. Einfacher zu implementieren, aber blockierend.
        \item \textbf{Asynchron:} Der Sender wartet nicht auf eine Antwort. Komplexer, aber performanter und nicht blockierend.
    \end{itemize}
\end{frame}

\begin{frame}{Kommunikationsprotokolle: TCP/IP, MQTT, HTTP}
    Die Wahl eines Protokolls hängt von den Anforderungen ab:
    \begin{itemize}
        \item \textbf{TCP/IP:} Zuverlässig, aber langsamer.
        \item \textbf{MQTT:} Leichtgewichtig, ideal für IoT.
        \item \textbf{HTTP:} Zustandslos, Basis für Webanwendungen.
    \end{itemize}
\end{frame}

\begin{frame}{RESTful APIs und Ressourcen}
    RESTful APIs verwenden Ressourcen als zentrale Elemente:
    \begin{itemize}
        \item Jede Ressource hat eine eindeutige URI.
        \item Ressourcen können über HTTP-Methoden (GET, POST, PUT, DELETE) manipuliert werden.
    \end{itemize}
    Vergleichbar mit einem Online-Shop: Produkte, Kunden und Bestellungen sind Ressourcen, die über die API verwaltet werden.
\end{frame}

\begin{frame}{HATEOAS (Hypermedia as the Engine of Application State)}
    HATEOAS ermöglicht es Clients, eine API dynamisch zu erkunden:
    \begin{itemize}
        \item Links zu verwandten Ressourcen sind in Antworten enthalten.
    \end{itemize}
    Vergleichbar mit einem interaktiven Buch: Links führen den Leser zu anderen Kapiteln.
\end{frame}

\begin{frame}{Richardson Maturity Model}
    Das Richardson Maturity Model beschreibt die Reife von RESTful APIs:
    \begin{itemize}
        \item \textbf{Level 0:} Einzelner URI und eine einzige HTTP-Methode (RPC-Stil).
        \item \textbf{Level 1:} Ressourcenorientierung, aber nicht alle HTTP-Methoden.
        \item \textbf{Level 2:} Verwendung aller HTTP-Methoden.
        \item \textbf{Level 3:} HATEOAS.
    \end{itemize}
    Höhere Levels stehen für "RESTvollere" APIs.
\end{frame}

\begin{frame}{Alternative APIs: CoAP, XMPP, AMQP, DDS, OPC-UA}
    Neben RESTful APIs existieren weitere Protokolle für spezifische Anwendungsfälle:
    \begin{itemize}
        \item \textbf{CoAP:} Für ressourcenbeschränkte Geräte.
        \item \textbf{XMPP:} Für Instant Messaging.
        \item \textbf{AMQP:} Für Enterprise Messaging.
        \item \textbf{DDS:} Für Echtzeitdaten.
        \item \textbf{OPC-UA:} Für industrielle Kommunikation.
    \end{itemize}
\end{frame}

\begin{frame}{Message Broker - MQTT vs. Kafka}
    \begin{itemize}
        \item \textbf{MQTT:} Leichtgewichtig, Publish-Subscribe, ideal für IoT.
        \item \textbf{Kafka:} Hochperformant, Streaming-Plattform, geeignet für große Datenmengen.
    \end{itemize}
    Die Wahl hängt vom Anwendungsfall ab.
\end{frame}

\begin{frame}{Message Broker Architektur}
    Ein Message Broker vermittelt die Kommunikation zwischen Anwendungen. Komponenten:
    \begin{itemize}
        \item \textbf{Broker:} Empfang, Speicherung und Weiterleitung von Nachrichten.
        \item \textbf{Producer:} Sendet Nachrichten an den Broker.
        \item \textbf{Consumer:} Empfängt Nachrichten vom Broker.
    \end{itemize}
    \textbf{Beispiele:} RabbitMQ, Kafka.
\end{frame}

\begin{frame}{Fehlersemantik: QoS in MQTT}
    MQTT bietet verschiedene Quality of Service (QoS) Levels:
    \begin{itemize}
        \item \textbf{QoS 0 (At most once):} Keine Garantie für die Zustellung.
        \item \textbf{QoS 1 (At least once):} Mindestens einmal Zustellung, aber Duplikate möglich.
        \item \textbf{QoS 2 (Exactly once):} Genau einmal Zustellung.
    \end{itemize}
    Die Wahl hängt von den Anwendungsanforderungen ab.
\end{frame}

\begin{frame}{Das Zwei-Generäle-Problem}
    Das Zwei-Generäle-Problem zeigt die Schwierigkeiten der zuverlässigen Kommunikation über unsichere Kanäle. Selbst mit Bestätigungen kann keine absolute Zustellgarantie erreicht werden. Dieses Problem ist zentral in verteilten Systemen.
\end{frame}

\begin{frame}{Serialisierungsformate}
    Serialisierungsformate wandeln Datenstrukturen in übertragbare Formate um:
    \begin{itemize}
        \item \textbf{Protocol Buffers:} Effizient, binär.
        \item \textbf{MessagePack:} Kompakt, effizient.
        \item \textbf{JSON:} Menschenlesbar, größer.
        \item \textbf{XML:} Flexibel, komplexer.
    \end{itemize}
    Die Wahl hängt von den Anforderungen ab.
\end{frame}

\begin{frame}{Koordination in verteilten Systemen}
    Koordination ist essenziell für die Zusammenarbeit von Prozessen. Techniken:
    \begin{itemize}
        \item \textbf{Snapshots:} Erfassen des globalen Systemzustands.
        \item \textbf{Checkpoints:} Speichern des Zustands zur Wiederherstellung.
        \item \textbf{Deadlock Detection:} Erkennen von Deadlocks.
        \item \textbf{Termination Detection:} Feststellen, wann alle Prozesse abgeschlossen sind.
    \end{itemize}
\end{frame}

\end{document}