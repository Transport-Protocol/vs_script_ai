\documentclass{beamer}

\usetheme{CambridgeUS}
\usecolortheme{dolphin}

\title{Verteilte Systeme}
\subtitle{Beispielarchitekturen und Realisierung}
\author{Prof. Dr. Martin Becke}
\date{\today}

\begin{document}

\begin{frame}
    \titlepage
\end{frame}

\begin{frame}{Beispielarchitekturen: RPC}
    Remote Procedure Call (RPC) ermöglicht das Aufrufen von Funktionen auf entfernten Systemen, als wären sie lokal. 
    Vergleichbar mit einem Telefonanruf: Der Anrufer (Client) tätigt einen Anruf (RPC) an den Angerufenen (Server), der den Anruf entgegennimmt und die gewünschte Aktion ausführt.
\end{frame}

\begin{frame}{RPC - Architektur}
    Die RPC-Architektur besteht aus:
    \begin{itemize}
        \item \textbf{Client und Server:} Zwei entkoppelte Kommunikationspartner.
        \item \textbf{Client- und Server-Stub:} Verwalten die Kommunikation und Datenkonvertierung (Marshalling).
        \item \textbf{Kommunikationsprotokoll:} Beispielsweise TCP/IP.
    \end{itemize}

    Der Client-Stub marshallt die Parameter, sendet die Anfrage an den Server, der die Funktion ausführt. Der Server-Stub marshallt das Ergebnis und sendet es zurück an den Client.
\end{frame}

\begin{frame}{RPC - Schichtenmodell und Marshalling}
    Im Schichtenmodell von RPC übernimmt die Middleware die Aufgabe der Kommunikation und Datenkonvertierung. \newline
    \textbf{Marshalling:} Transformation von Daten in ein Format, das über das Netzwerk übertragen werden kann.
\end{frame}

\begin{frame}{Marshalling vs. Serialisierung}
    Marshalling und Serialisierung sind verwandte Konzepte:
    \begin{itemize}
        \item \textbf{Marshalling:} Vorbereiten von Daten für RPC, inklusive Metadaten.
        \item \textbf{Serialisierung:} Umwandlung eines Objekts in eine lineare Darstellung, oft Teil des Marshalling-Prozesses.
    \end{itemize}
\end{frame}

\begin{frame}{Copy-in, Copy-out}
    \textbf{"Copy-in, Copy-out"} ist ein Prinzip beim Marshalling:
    \begin{itemize}
        \item \textbf{Copy-in:} Die Parameterwerte werden in eine Nachricht kopiert.
        \item \textbf{Copy-out:} Die Ergebnisse werden zurück in die ursprünglichen Variablen kopiert.
    \end{itemize}
    Dies vermeidet direkte Speichermanipulation zwischen Systemen.
\end{frame}

\begin{frame}{IDL und Codegenerierung}
    \textbf{Interface Definition Language (IDL)} beschreibt die Schnittstelle von entfernten Prozeduren. \newline
    Codegeneratoren erstellen automatisch Stub- und Skeleton-Code aus der IDL, was die Implementierung von RPC vereinfacht.
\end{frame}

\begin{frame}{Beispiel: XML-RPC und gRPC}
    \begin{itemize}
        \item \textbf{XML-RPC:} Verwendet XML für die Datenkodierung und HTTP für die Übertragung.
        \item \textbf{gRPC:} Verwendet Protocol Buffers (effizientes binäres Format) und HTTP/2.
    \end{itemize}
    Beide sind Beispiele für RPC-Implementierungen mit \textbf{"Copy-in, Copy-out"}-Semantik.
\end{frame}

\begin{frame}{Remote Method Invocation (RMI)}
    RMI ist ähnlich wie RPC, jedoch speziell für objektorientierte Sprachen wie Java. Es ermöglicht das Aufrufen von Methoden auf entfernten Objekten. \newline
    Vergleichbar mit RPC, jedoch mit Objekten statt Prozeduren.
\end{frame}

\begin{frame}{Beispielarchitekturen: P2P Chord}
    \textbf{Chord} ist ein P2P-System, das eine verteilte Hashtabelle implementiert. Es ermöglicht effizientes Suchen und Speichern von Daten in einem dezentralen Netzwerk. \newline
    Vergleichbar mit einem verteilten Telefonbuch: Jeder Knoten kennt nur einen Teil des Telefonbuchs, kann jedoch jeden Eintrag effizient finden.
\end{frame}

\begin{frame}{Chord - Struktur und Funktionsweise}
    \begin{itemize}
        \item \textbf{Identifier Space:} Ein Ring mit eindeutigen IDs für Knoten und Schlüssel.
        \item \textbf{Key-Node Mapping:} Schlüssel werden dem Knoten zugewiesen, dessen ID am nächsten liegt.
        \item \textbf{Finger Tables:} Ermöglichen effizientes Routing im Netzwerk.
    \end{itemize}
\end{frame}

\begin{frame}{Chord - Dynamik und Fehlertoleranz}
    Chord ist dynamisch und fehlertolerant:
    \begin{itemize}
        \item Knoten können hinzugefügt oder entfernt werden, ohne das System zu beeinträchtigen.
        \item Redundante Informationsspeicherung und regelmäßige Aktualisierungen der Finger Tables sorgen für Robustheit.
    \end{itemize}
\end{frame}

\begin{frame}{Cassandra - Anpassungen an Chord}
    Apache Cassandra, ein verteiltes NoSQL-Datenbanksystem, nutzt Ideen von Chord, jedoch mit Anpassungen:
    \begin{itemize}
        \item \textbf{Partitionierungsstrategie:} Cassandra verwendet einen Partitionsschlüssel zur Verteilung der Daten auf Knoten.
        \item \textbf{Replikationsstrategie:} Daten werden auf mehreren Knoten repliziert, um Verfügbarkeit und Fehlertoleranz zu gewährleisten.
        \item \textbf{Gossip-Protokoll:} Dient der Kommunikation zwischen Knoten.
    \end{itemize}
\end{frame}

\begin{frame}{Cassandra - Weiterentwicklung}
    Zusätzliche Techniken in Cassandra:
    \begin{itemize}
        \item \textbf{Lese- und Schreibkonsistenz:} Anpassbar an die Anforderungen der Anwendung.
        \item \textbf{Hinted Handoff:} Für den Umgang mit vorübergehend ausgefallenen Knoten.
        \item \textbf{Repair und Anti-Entropy:} Zur Erkennung und Behebung von Inkonsistenzen zwischen Replikaten.
    \end{itemize}
\end{frame}

\begin{frame}{Anti-Entropy}
    Anti-Entropy-Mechanismen stellen die Konsistenz zwischen Replikaten in verteilten Systemen sicher. \newline
    Sie erkennen und beheben Inkonsistenzen, die durch Updates, Netzwerklatenzen oder Ausfälle entstehen können. \newline
    Vergleichbar mit einem Reparaturdienst: Er stellt sicher, dass alle Kopien eines Dokuments identisch sind, auch wenn Änderungen an einer Kopie vorgenommen wurden.
\end{frame}

\end{document}