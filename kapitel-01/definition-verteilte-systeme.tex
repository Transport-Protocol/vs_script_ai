\subsection{Definition}

Eine Definition~\cite{coulouris2012distributed} von verteilten Systemen kann man als eine Gruppe von vernetzten, autonomen Computern verstehen, die gemeinsam eine Aufgabe erfüllen, ohne dass ein zentraler Koordinator benötigt wird. Dabei können die Computer geographisch verteilt sein und über unterschiedliche Hardware- und Software-Plattformen verfügen.

\importantvs{Laut Tanenbaum\cite{tanenbaum2017distributed} sind verteilte Systeme \enquote{eine Sammlung unabhängiger Computer, die den Benutzern als ein kohärentes System erscheinen} (Tanenbaum and van Steen, 2017, S. 4). Die Kommunikation in verteilten Systemen erfolgt über das Netzwerk und erfordert eine koordinierte Zusammenarbeit der einzelnen Komponenten.}
\mbox{}\\
Die Definition von einem Verteilten System ist aber nicht trennscharf. Daher existieren durchaus alternative Definitionen, denen man sich im Zweifelsfall gewahr werden muss. So können beispielhaft weitere Alternativen genannt werden, die die vorhandene Definition erweitern: \\
So definiert \cite{garg2016distributed} ein verteiltes System als mehrere unabhängige Computer, die miteinander kommunizieren und \enquote{kooperieren}, um gemeinsam eine Aufgabe zu erfüllen.
\\
Anders sieht es \cite{mukherjee2015distributed}, wo der Fokus mehr auf die technologische Komponente gelegt wird. Hier ist ein verteiltes System \enquote{ein Netzwerk} aus autonomen Computern, die miteinander kommunizieren und koordiniert zusammenarbeiten, um eine gemeinsame Aufgabe zu erfüllen.
\\\\
Sicher ist, es gibt viele verschiedene Definitionen von verteilten Systemen, die sich je nach Autor, Kontext und Zweck stark oder leicht unterscheiden können. Die grundlegende Idee hinter allen Definitionen ist jedoch die gleiche: Es geht um die Vernetzung und Zusammenarbeit unabhängiger Recheneinheiten.
\\\\
Nun kann man anführen, dass bei einer Vielzahl von Definitionen diese ihre Bedeutung verlieren, da ein effizienter und sinnvoller Einsatz in einer praktischen Umsetzung nicht hilfreich erscheint, doch ist hier folgendes anzuführen: 

Verteilte Systeme bietet generische Lösungsansätze für unterschiedlichen Problemklassen an. Da es sich bei Verteilten Systemen nicht selten um komplexe Systeme handelt, welche aus einer Vielzahl von miteinander kommunizierenden und kooperierenden Komponenten bestehen, haben die Abweichungen der Eigenschaften einen hohen Einfluss auf die Etablierung der generischen Lösungsansätze. Somit muss die doch meist sehr generische Definition um einzelne Eigenschaften angereichert werden, damit entsprechende Logikzweige für eine adäquate Lösung formuliert werden können. Die generische Definition ist daher als Ausgangspunkt zu sehen, für den weiteren Anforderungsprozess. Noch einen Schritt weiter, ist es besonders fordernd für den Anforderungsprozess, dass diese Eigenschaften der Komponenten auf verschiedenen Geräten und Standorten verteilt sein können und zunächst systematisch erfasst und vollständig beschrieben werden müssen. Fehler in diesem Prozess können weitreichende Folgen für den Erfolg der angestrebten Lösung haben. 
\\\\
Um die Herausforderung zu verdeutlichen, soll die Aufgabe der Wegesuche in einem Labyrinth herangezogen werden. Die Aufgabe kann sehr einfach nachvollzogen werden, wenn man folgenden Algorithmus, auch bekannt als Breitensuche, etabliert. 

\begin{itemize}
\item Legen Sie eine Warteschlange an und fügen Sie den Startknoten hinzu.

\item Solange die Warteschlange nicht leer ist, nehmen Sie den Knoten am Anfang der Warteschlange und prüfen Sie, ob es der Zielknoten ist. Wenn ja, haben Sie den Pfad zum Ziel gefunden und können den Algorithmus beenden.

\item Andernfalls fügen sie alle unbesuchten Nachbarknoten des aktuellen Knotens zur Warteschlange hinzu und markieren sie als besucht.

\item Wiederholen Sie Schritt 2 und 3, bis Sie den Zielknoten gefunden haben oder alle Knoten besucht wurden.
\end{itemize}
Diese Lösung ist trivial und durchläuft das Labyrinth schrittweise, indem der Algorithmus zuerst alle Knoten in der Nähe des Startknotens besucht, dann alle Knoten in der Nähe dieser Knoten, und so weiter, bis er den Zielknoten erreicht oder alle Knoten besucht wurden.

Die Herausforderung in diesem Fall ist nicht das Verständnis für den Algorithmus, sondern die Herausforderung beginnt im Anforderungsprozess des verteilten Systems. Es ist wichtig zu beachten, dass der vorgestellte Lösungsansatz nur funktioniert, da das Labyrinth als Graph modelliert werden konnte, und der Informatiker dies in der Anforderungsanalyse erkennt. Wobei im verteilten System die Knoten als Positionen im Labyrinth und die Kanten, als  möglichen Wege, willkürlich verteilt sein können und somit die Herausforderung darin besteht, dass das ganze Bild des Labyrinths nicht direkt ersichtlich ist. So ist aber für eine adäquate Lösungsfindung genau diese Identifikation wesentlich, damit der mit der Lösung verbundene Graph und die damit verbundene Breitensuche als Logikbaum für eine effiziente Lösung identifiziert werden kann. 
\\\\
Die Bandbreite von Anwendungen, die verteilte Systeme nutzen, ist enorm und reicht von großen Online-Transaktionssystemen bis hin zu drahtlosen Sensornetzwerken, kurz es ist häufig sehr viel komplexer als die Suche nach einem Ausgang in einem Labyrinth oder die Identifikation einer einzelnen Problemklasse. Daher ist es wichtig, eine Definition und eine klare Abgrenzung zu schaffen, um sicherzustellen, dass die Systeme effektiv entworfen, entwickelt und implementiert werden können. Hier werden jedoch oft Fehler gemacht, da nicht selten generische Ansätze ohne Berücksichtigung des Anwendungsfalls diskutiert werden und somit möglicherweise nicht zur angestrebten Lösung passen, da sie die notwendigen Charaktereigenschaften nicht repräsentieren. Somit ist die Erweiterung der Definition des vorhandenen verteilten Systems grundsätzlich eine Aufgabe die zu erfüllen ist, unabhängig des Startpunktes der gewählten generischen Definition.
\\\\
Die Untersuchung der Eigenschaften von verteilten Systemen ist auch für die Forschung und Entwicklung noch immer sehr relevant.
\\\\
Dennoch, am Anfang braucht man für den gesamten Prozess einen Startpunkt. Der Informatiker muss sich bewusst machen, dass die zu lösende Aufgabe ein verteiltes System repräsentiert. Für eine erste Einschätzung der Situation, ob das Themengebiet der verteilten Systeme zum Tragen kommt, genügt aber ein kleiner Schritt. Dieser Schritt umfasst die Prüfung einer wesentlichen Eigenschaft, die alle Definitionen miteinander verbindet.  Kurz gesprochen ist der gemeinsame Nenner das \enquote{nicht-vorhanden-sein} eines gemeinsamen (Arbeits-)Speichers. Wird kein gemeinsamer Speicher genutzt, um sich zu koordinieren, ist die Umsetzung eines verteilten Systems sehr wahrscheinlich und der Startpunkt gegeben, um die Anforderungen an das System und damit die genutzte Definition entsprechend zu verfeinern.
