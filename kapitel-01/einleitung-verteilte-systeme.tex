\subsection{Verteilte Systeme in Hamburg}

Die Stadt Hamburg ist ein wichtiger Wirtschaftsstandort in Deutschland und beherbergt zahlreiche Unternehmen, die auf innovative Technologien und Dienstleistungen angewiesen sind. In der Informatik haben sich verteilte Systeme als zentrales Konzept etabliert, um Innovationen	 mit komplexen Anwendungen und Dienste zu realisieren, insbesondere wenn diese über mehrere Standorte oder Rechenzentren verteilt sind.
\\\\
In Hamburg werden verteilte Systeme in verschiedenen Branchen eingesetzt, wie z.B. im Finanzsektor, im Gesundheitswesen, in der Logistik oder im öffentlichen Sektor. Dieser Text soll ein paar ausgewählte Beispiele aufzeigen.
\\\\
Der Hamburger Hafen ist einer der größten Häfen Europas und spielt eine wichtige Rolle im internationalen Handel. Mit dem Konzept des Smart Ports wird der Hafen als modernes und integriertes System betrachtet, das auf der Verwendung von verteilten Systemen und damit umgesetzte Prozesse basiert.
Ein Smart Port ist ein System, das verschiedene Technologien und Konzepte integriert, um den Hafen effizienter, sicherer und nachhaltiger zu machen. Dazu gehören unter anderem Sensoren, Big-Data-Analysen, Cloud-Computing-Systeme und mobile Anwendungen. Alle diese Technogien können unter dem Aspekt Verteilte Systeme zusammengefasst werden.  
\\\\
Der Smart Port in Hamburg\cite{HPA} wird von verschiedenen Partnern betrieben, wie z.B. der Hamburg Port Authority (HPA), Hafenunternehmen und IT-Dienst-leistern. Die verteilten Systeme, die im Smart Port eingesetzt werden, ermöglichen es, Daten und Informationen in Echtzeit zu sammeln, zu analysieren und zu teilen. Dadurch können die Betriebsabläufe im Hafen optimiert und die Effizienz gesteigert werden.
\\\\
Ein weiteres Beispiel für die Verwendung von verteilten Systemen im Smart Port ist das Hafenverwaltungssystem, das von der HPA betrieben wird. Das System besteht aus verschiedenen Modulen, die es ermöglichen, den Schiffsverkehr und die Lagerung von Waren im Hafen zu verwalten. Dazu gehören auch Funktionen zur Planung von Liegeplätzen und zur Überwachung von Sicherheitsaspekten.
Technologien, die im Smart Port eingesetzt werden, sind u.a. Automatisierungstechnologien für Containerterminals und die Verwendung von Blockchain-Technologie zur Verbesserung der Sicherheit und Effizienz von Lieferketten. Alles dies sind wesentliche Technologien aus derm Kern der Verteilten Systeme.

Der Smart Port in Hamburg zeigt auf besondere Weise, wie verteilte Systeme mit weiteren innovative Technologien aus Maschinenbau oder Werkzeugbau  eingesetzt werden können\cite{Bockenfeld2020}, um komplexe Systeme wie einen Hafen zu optimieren und nachhaltiger zu gestalten.
\\\\
Auch die Luftfahrtindustrie, wie Airbus, setzt eine Vielzahl von verteilten Systemen ein, um Flugzeuge sicher und effizient zu betreiben. Ein verteiltes System in einem Flugzeug umfasst eine Vielzahl von Technologien, die es ermöglichen, verschiedene Komponenten und Systeme miteinander zu vernetzen, um eine reibungslose Kommunikation und Zusammenarbeit zu gewährleisten\cite{chen2017distributed}. Verteilte Systeme in Flugzeugen umfassen eine Vielzahl von Systemen, wie z.B. Navigationssysteme, Autopiloten, Flugsicherheitssysteme und Flugüberwachungssysteme. Diese Systeme kommunizieren miteinander über verschiedene Netzwerke, wie z.B. CAN-Bus, Ethernet und WLAN und müssen zeitlich und örtlich koordiniert werden, wie auch internationalen Anforderungen entsprechen.

Die Verwendung von verteilten Systemen in der Luftfahrtindustrie hat zahlreiche Vorteile, wie z.B. die Verbesserung der Flugsicherheit, die Steigerung der Effizienz und die Reduzierung von Ausfallzeiten. Allerdings gibt es auch Herausforderungen, wie z.B. die Notwendigkeit einer sorgfältigen Planung und Integration sowie die Gewährleistung der Datensicherheit und des Datenschutzes. Ein Beispiel für ein verteiltes System in einem Flugzeug ist das Fly-by-Wire-System. Das Fly-by-Wire-System ist ein elektronisches Steuerungssystem, das den mechanischen Steuermechanismus durch eine elektronische Steuerung ersetzt\cite{tian2017fault}. Das System besteht aus mehreren Komponenten, die miteinander kommunizieren und koordinieren, um das Flugzeug zu steuern. Die Komponenten sind über verschiedene Netzwerke, wie z.B. Ethernet\cite{wang2018architecture}, miteinander verbunden.

Weitere Technologien, die in verteilten Systemen in Flugzeugen eingesetzt werden, sind u.a. Avionik-Systeme, die Daten zur Flugsicherheit sammeln und verarbeiten, oder auch WLAN-Systeme, die eine schnelle und zuverlässige drahtlose Verbindung zwischen den Komponenten des Flugzeugs ermöglichen.
\\\\
Ein weiteres wichtiges Standbein für Hamburg ist der Kaufmanns-, Finanz- und Versicherungssektor. Hier spielen verteilte Systeme eine außergewöhnliche und immer stärkere Rolle. Ein Beispiel für ein verteiltes System im Finanzwesen ist das Real-Time Gross Settlement System (RTGS)\cite{rtgs}, das von Zentralbanken auf der ganzen Welt eingesetzt wird. Dieses System ermöglicht die sofortige Übertragung von großen Geldbeträgen zwischen Banken und anderen Finanzinstitutionen in Echtzeit. RTGS nutzt eine verteilte Datenbank, um alle Transaktionen aufzuzeichnen und sicherzustellen, dass alle Parteien die gleichen Informationen haben.

Auch das Clearinghaus-System\cite{clearinghouse}, das von Finanzinstitutionen und Versicherungen genutzt wird, um Transaktionen abzuwickeln und Zahlungen zu verarbeiten ist erwähnenswert. Das System nutzt eine verteilte Datenbank, um alle relevanten Informationen zu speichern und zu verarbeiten, um sicherzustellen, dass alle Parteien korrekt abgerechnet werden.
\\
Ein Zeitgeist Thema sind Blockchain-Systeme\cite{blockchain}, welche in der Logisitk-, Finanz- und Versicherungsbranche immer beliebter werden. Blockchain-Systeme nutzen eine verteilte Datenbank, um Transaktionen aufzuzeichnen und sicherzustellen, dass sie unveränderlich sind. Diese Technologie wird bereits in der Kryptowährungsbranche eingesetzt und findet auch in anderen Bereichen wie der Versicherung und der Verwaltung von Vermögenswerten Anwendung.
\\\\
Dies ist natürlich nur eine sehr kleine Auswahl von Beispielen und kann sicher um eine Vielzahl anderer Beispiele ergänzt werden. 