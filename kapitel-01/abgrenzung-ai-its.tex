\subsection{Abgrenzung}
Eine Abgrenzung des Begriffs \enquote{Verteilte Systeme} in der Informatik kann nur auf unterschiedlichen Ebenen gelingen, da es sich bei Verteilten Systemen um ein sehr breites und vielschichtiges Feld handelt. Hier soll auch kein Versuch unternommen dies vollständig zu tuen, sondern es soll nur einen Blick in diese Diskussion geben. 
\\\\
Die Abgrenzung des Begriffs hängt davon ab, auf welcher Ebene man das verteilte System betrachtet und welche spezifischen Merkmale und Eigenschaften man berücksichtigen möchte. Eine klare Abgrenzung ist - wie bereits ausgeführt - wichtig, um sicherzustellen, dass die Systeme effektiv entworfen, entwickelt und implementiert werden können.

\subsubsection{Verteiltes und monolithisches System}

Auch wenn dieses Dokument sich primär auf die Umsetzung von verteilten Systemen konzentriert, darf nicht vergessen werden, dass dieser nur ein Kontext von vielen ist. Ein sehr bekanntes Architekturmodell welches als Alternative zum Verteilten System verstehen kann, ist das monolithische System\cite{Newman2015}. 
\\\\
Ein monolithisches System mit Fokus auf die Anwendung\cite{Szyperski2002} ist eine Softwarearchitektur, bei der alle Komponenten einer Anwendung in einer einzigen, großen Codebase zusammengefasst sind. Es gibt keine klare Trennung zwischen den einzelnen Komponenten, sondern alle Funktionen und Prozesse laufen innerhalb des gleichen Prozesses und können auf gemeinsame Ressourcen zugreifen. Im Gegensatz zu Verteilten Systemen gibt es keine Notwendigkeit, die Kommunikation zwischen verschiedenen Rechnern oder Prozessen zu koordinieren, da alles in einem einzigen Prozess läuft. Monolithische Systeme waren lange Zeit die Standardarchitektur für Softwareanwendungen, bevor Verteilte Systeme aufkamen.
\\\\
Ein monolithisches System ist in der Regel einfacher zu entwickeln, zu testen und zu deployen als ein Verteiltes System. Es ist auch einfacher, die Performance und Skalierbarkeit des Systems zu optimieren, da alles in einer einzigen Anwendung läuft. Außerdem gibt es weniger komplexe Interaktionen und weniger Schnittstellen zwischen den Komponenten, was die Wartung und das Debugging erleichtert. Dies sind klare Argumente für die Wahl der Architektur in der Praxis, wenn eine Wahl besteht. 
\\\\
Ein weiterer Vorteil eines monolithischen Systems ist, dass es in der Regel eine bessere Sicherheit und Zuverlässigkeit bietet, da es einfacher ist, das System als Ganzes zu überwachen und zu sichern. Bei Verteilten Systemen kann es schwieriger sein, alle Komponenten zu überwachen und zu sichern, da sie möglicherweise über verschiedene Netzwerke und Standorte verteilt sind.
\\\\
Allerdings hat ein monolithisches System auch einige Nachteile. Es kann schwieriger sein, neue Technologien und Tools zu integrieren, da das gesamte System angepasst werden muss. Außerdem kann es schwieriger sein, das System auf verschiedene Umgebungen zu skalieren und zu verteilen, da es als Ganzes betrachtet wird.
\\\\
Als ein klassischer extremer Gegenentwurf aus Sicht der Softwarearchitektur zum monolithischen System kann aus dem Kontext der Verteilten Systeme auch die Microservice-Architektur\cite{fowler2014microservices} betrachtet werden.
\\\\
Eine Microservice-Architektur ist eine spezielle Form der Verteilten Systeme, bei der eine Anwendung in einzelne unabhängige Komponenten (Services) aufgeteilt wird. Jeder Service ist in der Regel für eine bestimmte Funktion verantwortlich und kann separat entwickelt, getestet, bereitgestellt und skaliert werden. Die Services kommunizieren über definierte Schnittstellen miteinander und werden oft in Containern bereitgestellt.
\\\\
Somit kann man sagen, dass eine Microservice-Architektur ein Verteiltes System darstellt, da die Services auf verschiedenen Servern oder in verschiedenen Containern laufen und über das Netzwerk miteinander kommunizieren. Die Vorteile eines Verteilten Systems, wie Skalierbarkeit, Ausfallsicherheit und Flexibilität, gelten auch für eine Microservice-Architektur. Allerdings bringt die Komplexität der Kommunikation zwischen den Services auch zusätzliche Herausforderungen mit sich, wie z.B. die Sicherstellung der Konsistenz und Verfügbarkeit der Daten. Diese Herausforderungen hat ein monolithisches System nicht.


\subsubsection{Verwandte Großrechner/Mainframes}
Ein Großrechner ist ein leistungsstarkes Computersystem, das typischerweise in einem Rechenzentrum betrieben wird und für die Verarbeitung von großen Datenmengen und rechenintensiven Aufgaben wie Simulationen, wissenschaftlichen Berechnungen oder Geschäftsanwendungen verwendet wird. Ein Großrechner zeichnet sich durch seine hohe Leistung, Zuverlässigkeit, Verfügbarkeit und Sicherheit aus. Großrechner werden oft in großen Unternehmen, Regierungsbehörden oder Forschungseinrichtungen eingesetzt.
\\\\
Im Gegensatz zu Großrechnern sind verteilte Systeme ein Netzwerk von miteinander verbundenen Computern, die koordiniert arbeiten, um eine gemeinsame Aufgabe zu erfüllen. Verteilte Systeme sind darauf ausgelegt, Last und Ressourcen auf mehrere Computer zu verteilen und können von kleinen Netzwerken bis hin zu globalen Netzwerken von Millionen von Computern reichen. Verteilte Systeme können für verschiedene Anwendungen eingesetzt werden, darunter verteilte Datenbanken, verteilte Rechen- und Speicherressourcen, Webanwendungen und Cloud Computing.
\\\\
Der Hauptunterschied zwischen einem Großrechner und einem verteilten System liegt in der Art und Weise, wie die Ressourcen bereitgestellt werden. Ein Großrechner verfügt über eine leistungsstarke Architektur und eine hohe Anzahl von Speicher- und Ein-/Ausgabe-Subsystemen, die mit Hochgeschwindigkeitsverbindungen verbunden sind. Ein verteiltes System hingegen besteht aus vielen Computern, die miteinander verbunden sind und zusammenarbeiten, um eine Aufgabe zu erfüllen.
\\\\
Ein weiterer Unterschied besteht in der Art und Weise, wie die Ressourcen verwaltet werden. Ein Großrechner verfügt normalerweise über einen dedizierten Systemadministrator, der für die Verwaltung der Hardware und Software des Systems verantwortlich ist. Verteilte Systeme hingegen sind oft dezentralisiert und verfügen über eine gemeinsame Management-Infrastruktur, die die Ressourcen überwacht und verwaltet.

\subsubsection{Abstraktionsebenen}
Eine klare Definition von Verteilten Systemen fällt insbesondere auch daher schwer, da Verteilte Systeme nicht immer eine vergleichbare abstrahierte Struktur aufweisen. Stattdessen können verschiedene Abstraktionsansätze auf unterschiedlichen Ebenen für die Analyse eines Verteilten Systems angenommen werden. Um diese unterschiedlichen Ebenen zu verdeutlichen, soll zunächst die Unterscheidung in zwei Ebenen angedacht werden, auch wenn eine Vielzahl anderer Einteilungen angenommen werden können. 
\\\\
So können verteilte Systeme in einem ersten trivialen Ansatz von einer technologischen und einer anwendungsbezogenen Ebene betrachtet werden\cite{coulouris2012distributed}. Auf der technologischen Ebene werden Verteilte Systeme als eine Sammlung von Computern und Netzwerken interpretiert, die miteinander kommunizieren und kooperieren, um eine gemeinsame Aufgabe zu erfüllen. Auf der anwendungsbezogenen Ebene geht es dagegen um die Verwendung von Verteilten Systemen zur Lösung von Problemen in bestimmten Anwendungsdomänen.
\\\\
Ein Beispiel für ein verteiltes System auf technologischer Ebene ist das Internet. Das Internet besteht aus einer Vielzahl von Computern und Netzwerken, die miteinander verbunden sind und Informationen austauschen. Andere Beispiele für verteilte Systeme auf technologischer Ebene sind Cloud-Computing-Plattformen, Peer-to-Peer-Netzwerke und verteilte Datenbanken.
\\\\
Auf der anwendungsbezogenen Ebene gibt es ebenfalls eine Vielzahl von Anwendungen, die auf Verteilten Systemen basieren. Ein Beispiel hierfür sind Social-Media-Plattformen wie Facebook und Twitter, die auf einer verteilten Architektur aufgebaut sind und es Benutzern ermöglichen, Informationen und Nachrichten auszutauschen. Andere Beispiele für Verteilte Systeme auf anwendungsbezogener Ebene sind E-Commerce-Plattformen oder Anwendungen im Gesundheitswesen.
\\\\
Da dieses Dokument primär in der Lehre eingesetzt werden soll, wird der Begriff für die Angewandte Informatik (AI) und der Informatik für Technische Systeme (ITS) diskutiert. 

\subsubsection{Verteilte Systeme in AI und ITS}

Die Unterscheidung zwischen Verteilten Systemen in der eher technischen Informatik und in der angewandten Informatik ist nicht immer klar abzugrenzen, da beide Bereiche sich oft überschneiden. Es gibt jedoch einige Unterschiede, die man beachten kann.

In der technischen Informatik geht es hauptsächlich um die Entwicklung von Technologien und Infrastrukturen, wie z.B. Netzwerkprotokollen und verteilten Datenbanken \cite{coulouris2012distributed}. Der Schwerpunkt liegt hier auf der Entwicklung von effizienten und zuverlässigen Algorithmen und Protokollen für verteilte Systeme. Auf der anderen Seite konzentriert sich die angewandte Informatik auf die Anwendung von verteilten Systemen in verschiedenen Anwendungsbereichen wie E-Commerce, E-Government, und Cloud Computing \cite{ghosh2012distributed}. Der Schwerpunkt liegt hier auf der Anwendung von vorhandenen Technologien und Infrastrukturen für Verteilte Systeme, um konkrete Probleme zu lösen.
\\\\
Ein weiterer nicht unerheblicher Unterschied besteht darin, dass Verteilte Systeme in der technischen Informatik oft auf die Forschung und Entwicklung von neuen Technologien und Algorithmen fokussiert sind, die in den Schnittstellen eine hohe Nähe zu Hardwareentwicklern aufweisen, während Verteilte Systeme in der angewandten Informatik oft eine größere Rolle bei der Umsetzung von Geschäftsprozessen spielen und somit nicht selten eng mit dem Management von Unternehmen verbunden sind\cite{birman2012guide}.

\subsubsection{Aspekte und Sichten}
Bisher lag der implizierte Fokus der Diskussion primär auf den Aspekten des Designs und der Architektur von Verteilten Systemen sowie der technologischen Basis, wie sie in Netzwerkprotokollen und -kommunikation zu finden ist. Darüber hinaus können jedoch auch andere Aspekte genannt werden, unter denen ein Verteiltes System diskutiert werden kann \cite{tanenbaum2017distributed}. 
\begin{itemize} 
\item Skalierbarkeit und Ausfallsicherheit: Verteilte Systeme sollen nicht selten in der Lage sein -  mit steigender Last, bei Ausfällen oder Fehlern - selbst in Teilen weiterhin zu funktionieren. Hierzu müssen spezifische Skalierungs- und Wiederherstellungsmechanismen implementiert werden.
\item Datenmanagement: In Verteilten Systemen müssen Daten effektiv verwaltet und zwischen verschiedenen Komponenten ausgetauscht werden. Hierbei müssen spezifische Datenmanagement- und -konsistenzmechanismen implementiert sein, die einen neuen Umgang mit Kosistenz erfordern.
\item Orchestrierung und Deployment: In Verteilten Systemen müssen die verschiedenen Komponenten orchestriert und deployed werden, um eine reibungslose Kommunikation und Zusammenarbeit zu gewährleisten. Hierbei müssen spezifische Orchestrierungs- und Deployment-Mechanismen definiert und implementiert werden. Auch das Testen bringt hier neue Herausforderungen.
\item Sicherheit: In Verteilten Systemen müssen Maßnahmen ergriffen werden, um die Sicherheit der übertragenen Daten und der beteiligten Systeme zu gewährleisten. Dazu gehören Authentifizierung, Verschlüsselung und Schutz vor Angriffen. Dies ist eine sehr wichtige Sicht auf das System, auch wenn es kein zentraler Fokus dieser Ausarbeitung ist.
\end{itemize}	
Alle Aspekte implizieren unterschiedliche Anforderungen, die disjunkt betrachtet werden können, aber nicht selten in einer Interaktion und Abwägung zueinander stehen, was eine weitere Komplexität für diesen bereits intensiven Entwicklungsprozess einbringt. Auch Aspekte aus der Organisation eines Entwicklungsteams selbst geben immer wieder neue Impulse, um die Herausforderungen von Verteilten Systemen neu zu interpretieren und zu diskutieren. Diese Disussionen geben gerne eine Einteilung in Rollen und Sichten vor. Ein Versuch ein Verteiltes System in Sichten einzuordnen, soll in der folgenden Aufzählung versucht werden:
\begin{itemize} 
\item Architektursicht: Hier liegt der Fokus auf der Struktur und Organisation des Systems sowie der Verteilung der Komponenten und deren Interaktion. Es geht darum, die Komponenten des Systems zu identifizieren, deren Aufgaben und Verantwortlichkeiten zu definieren und eine geeignete Architektur zu entwerfen, die eine effiziente Zusammenarbeit der Komponenten ermöglicht.
\item Prozesssicht: In der Prozesssicht steht die zeitliche Abfolge der Abläufe und Interaktionen im Vordergrund. Es geht darum, die Prozesse und Workflows im System zu identifizieren und zu modellieren, um eine effektive Koordination und Steuerung der Aktivitäten zu gewährleisten.
\item Datensicht: Die Datensicht fokussiert auf die Daten, die im System verarbeitet werden. Hier geht es um die Definition und Modellierung der Datenstrukturen, die in den Komponenten des verteilten Systems verwendet werden, sowie um die Speicherung, Verarbeitung und Übertragung der Daten.
\item Sicherheitssicht: Die Sicherheitssicht betrachtet die Aspekte der Datensicherheit und des Datenschutzes im Verteilten System. Es geht darum, Risiken und Bedrohungen zu identifizieren und geeignete Maßnahmen zu ergreifen, um die Daten und das System insgesamt vor unautorisiertem Zugriff, Missbrauch oder Diebstahl zu schützen.
\item Betriebssicht: Die Betriebssicht betrachtet Aspekte wie die Verfügbarkeit, Zuverlässigkeit und Skalierbarkeit des Verteilten Systems. Hier geht es darum, sicherzustellen, dass das System kontinuierlich und zuverlässig arbeitet und bei Bedarf flexibel skaliert werden kann.
\item Entwicklersicht: Die Entwicklersicht betrachtet Aspekte der Entwicklung und Implementierung von Verteilten Systemen. Hier geht es um die Auswahl geeigneter Programmiersprachen, Frameworks und Tools sowie um die Anwendung von Best Practices und Methoden, um eine effiziente und qualitativ hochwertige Entwicklung zu gewährleisten.
\end{itemize}	
Je nach Kontext und Zielsetzung können auch weitere Sichten auf ein Verteiltes System relevant sein, beispielsweise die ökonomische bzw. ökologische Sicht oder die Benutzersicht. Auch hier können die einzelnen Diskussionspunkte eine beliebige Komplexität annehmen. Wichtig ist, im Vorfeld einer Diskussion zu bestimmen welcher Fokus die Dikussion hat. Alles miteinander zu verweben, ist eine sehr komplexe herangehensweise, die nicht selten zum Scheitern verurteilt ist. Das Teile-und-herrsche Verfahren ist hier ein erster guter Ansatz um Arbeitpakete und Aufgabenfelder zu schnüren. 
\\\\
Als Beispiel soll die ökonomische/ökologische Anforderung an den Energieverbrauch von Transaktionen eingebracht werden. Es ist schwierig vorab eine generische aber genaue Antwort auf diese Frage zu geben, da dies von vielen Faktoren abhängt, wie z.B. der Art der Datenbank, der Größe und Komplexität der Transaktionen und der Energiequelle, die zur Unterstützung der Datenbank verwendet wird.
\\\\
Soll weiter die Anforderung an den \ce{CO2} Verbrauch orientiert werden, gibt es schon vieles alleine bei der Datenhaltung zu beachten. Beispielhaft gibt es einige Schätzungen und Studien, die einen Vergleich zwischen der Energieeffizienz von Datenbanktransaktionen und Blockchain-Transaktionen ermöglicht. Laut einer Studie von PwC Deutschland aus dem Jahr 2019\cite{pwc2019} benötigt eine durchschnittliche Bitcoin-Transaktion etwa 550 kWh, um verarbeitet zu werden. Zum Vergleich: Eine einzelne Visa-Transaktion benötigt nur 0,002 kWh. Dies bedeutet, dass Sie mit einer einzigen Blockchain-Transaktion Hunderte oder sogar Tausende von Datenbanktransaktionen durchführen könnten, um die gleiche Menge an Energie zu verbrauchen.
Es ist jedoch auch wichtig zu beachten, dass die Energieeffizienz von Blockchain-Transaktionen zunehmend verbessert wird, insbesondere durch den Einsatz von Proof-of-Stake-Konsensmechanismen anstelle des energieintensiven Proof-of-Work. Die Verfahren werden in dieser Ausarbeitung noch miteinander besprochen und diskutiert. Darüber hinaus können Datenbanken selbst auch energieeffizienter gestaltet werden, indem Anfragen umgebaut und gang generell die Energiequelle auf erneuerbare Energiequellen geändert wird. Auch ein Umstieg auf energieeffiziente Hardware ist denkbar. 
\\\\ 
Aber selbst wenn man mit dem Teile-und-herrsche Aspekt erfolgreich die Teilaspekte erfolgreich beleuchtet, darf man dabei das gesamte System nicht aus dem Auge verlieren. Die Sichten wirken aufeinander. Dies ist auch stark Erkennbar in moderneren Entwicklungsmethoden, wo versucht wird verschiedene Sichten natürlich im (Entwicklungs-)Prozess zu vereinen. Als Anwendungsbeispiel aus der Praxis kann die Entwicklersicht mit der Betriebssicht verbunden werden. Hier soll als Beispiel eine DevOps\cite{kumar2016devops} Variante genauer betrachet werden. 
\\\\
DevOps ist eine agile Methode, die darauf abzielt, die Zusammenarbeit zwischen der Entwicklung (Dev) und dem Betrieb (Ops) von Software zu verbessern, um schnellere und zuverlässigere Softwarebereitstellungen zu erreichen. Verteilte Systeme führen zu komplexeren Architekturen, die eine engere Zusammenarbeit zwischen den Entwicklern und Betriebsteams erfordern. In einem Verteilten System müssen Entwickler und Betriebsteams sich auf die gemeinsame Verantwortung für die Verfügbarkeit, Leistung und Sicherheit der Systeme konzentrieren. Dies erfordert eine enge Zusammenarbeit und Kommunikation zwischen den Teams, um sicherzustellen, dass alle Aspekte des Systems effektiv und effizient verwaltet werden. Aus der Methode lassen sich dann weitere Praktiken, Werkzeuge und Technologien ableiten. 
DevOps-Praktiken, wie Continuous Integration (CI) und Continuous Delivery (CD) sind heute wichtige Schlagwörter, wobei die technologischen Lösungen in den Kontexten nicht selten wieder ein Verteiltes System darstellen. Durch die Automatisierung von Entwicklungs- und Bereitstellungsprozessen können Entwickler und Betriebsteams schnell auf Änderungen im System reagieren und sicherstellen, dass neue Funktionen und Updates ohne Unterbrechung der Systemleistung bereitgestellt werden können. Dies ist insbesondere wichtig bei Verteilten Systemen, da die Komplexität der Systeme die manuelle Verwaltung erschwert und Zeit und Ressourcen erfordert. Dieser Ansatz kommt aber auch mit Nachteilen, wie zum Beispiel hohe Einarbeitungskosten, kulturelle Herausforderungen (Dev und Ops) oder auch die neue Komplexität bei der Verbindung der Sichten.
\\\\
Was am Ende aber bleibt ist, das die Entwicklung und Bereitstellung von Verteilten Systemen eine enge Zusammenarbeit aller Stakeholder mit ihren Sichen, sowie die Nutzung von  modernen Entwicklungspraktiken und Automatisierungstools, um sicherzustellen, dass die Systeme effektiv und effizient verwaltet werden.
