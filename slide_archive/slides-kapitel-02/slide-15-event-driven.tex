%\section{Einleitung}
\subsection{Event-Driven Architektur}
\begin{frame}
  \frametitle{Event-Driven Architecture}
  \framesubtitle{Idee}
  \begin{itemize}
    \item Kommunikation getrieben über Events oder Nachrichten
    \item Komponenten reagieren auf Events oder Nachrichten
    \item Keine direkte Kommunikation (Vergleich Eventbus)
  \end{itemize}
\end{frame}

\begin{frame}
  \frametitle{Event-Driven Architecture}
  \framesubtitle{Beispiele}
  \begin{itemize}
    \item Aktienhandelssystem
    \item IoT-Sensornetzwerk
    \item E-Commerce-Plattform
  \end{itemize}
\end{frame}

\begin{frame}
  \frametitle{Event-Driven Architecture}
  \framesubtitle{Aufbau}
  \begin{itemize}
    \item Event-Producer
    \item Event-Channel
    \item Event-Consumer
  \end{itemize}
\end{frame}

\begin{frame}
  \frametitle{Event-Driven Architecture}
  \framesubtitle{Erweiterung um Lambda Architecture}
  \begin{itemize}
    \item Häufig im Kontext von Big-Data- und Echtzeitanalysen
    \item Datenverarbeitungsarchitekturmuster
    \item Latenzarme und fehlertolerante Analyse- und Verarbeitungssysteme
  \end{itemize}
\end{frame}

\begin{frame}
  \frametitle{Lambda Architecture}
  \framesubtitle{Aufbau}
  \begin{itemize}
    \item Batch-Layer
    \item Speed-Layer
    \item Serving-Layer
  \end{itemize}
\end{frame}


