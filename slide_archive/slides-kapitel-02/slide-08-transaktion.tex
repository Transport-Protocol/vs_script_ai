%\section{Einleitung}
\subsection{Transaktion}
\begin{frame}
  \frametitle{Transaktion}
  \framesubtitle{Idee}
  \begin{itemize}
    \item Eine Sequenz von Operationen
    \item Wichtig für Konsistenz und Integrität
    \item ACID-Eigenschaften 
  \end{itemize}
\end{frame}
\begin{frame}
  \frametitle{Transaktion}
  \framesubtitle{Beispiele}
  \begin{itemize}
    \item Bankwesen
    \item E-Commerce
    \item Verteilte Datenbanken
    \item Roboter/Urlaub (Diskussion)
  \end{itemize}
\end{frame}
\begin{frame}
  \frametitle{Transaktion}
  \framesubtitle{Transaktionsmanager}
  \begin{itemize}
    \item Verschachtelte Transaktionen 
    \item Größere Transaktionen separat abschlie  ßen
    \item Koordiniert zurückzurollen
  \end{itemize}
\end{frame}

\begin{frame}
  \frametitle{Transaktionsmanager}
  \framesubtitle{Aufgaben}
  \begin{itemize}
    \item Koordination
    \item Protokollierung und Wiederherstellung
    \item Isolierung und Synchronisation
    \item Commit und Rollback
  \end{itemize}
\end{frame}

\begin{frame}
  \frametitle{Transaktion}
  \framesubtitle{Message Passing}
  \begin{itemize}
    \item Kommunikationsparadigma
    \item Grundlegendes Konzept
    \item Nutzt Nachrichten, die Daten oder Anweisungen enthalten
  \end{itemize}
\end{frame}

\begin{frame}
  \frametitle{Message Passing}
  \framesubtitle{Eigenschaften}
  \begin{itemize}
    \item Asynchrone Kommunikation
    \item Lose Kopplung
    \item Skalierbarkeit
  \end{itemize}
\end{frame}

\begin{frame}
  \frametitle{Message Passing}
  \framesubtitle{Beispiele}
  \begin{itemize}
    \item Message Queues
    \item Message Passing Interface (MPI)
    \item Publish-Subscribe-Systeme
    \item MOM
  \end{itemize}
\end{frame}

\begin{frame}
  \frametitle{Message Passing}
  \framesubtitle{Actor-Modell}
  \begin{itemize}
    \item Setzt auf das Konzept von Message Passing auf
    \item Ist ein Konzept für das Design von verteilten Systemen
    \item Actoren sind grundlegende Recheneinheiten
    \item Isolation
    \item Nachrichtenbasiert
    \item Lokalitätstransparent
    \item Fehlertolerant
    \item Beispiel CAF (HAW Hamburg)
  \end{itemize}
\end{frame}

\begin{frame}
  \frametitle{Idempotent}
  \framesubtitle{Eigenschaft}
  \begin{itemize}
    \item Operationen, die wiederholt ausgeführt werden können, ohne dass sich das Ergebnis nach der ersten Anwendung ändert
    \item Beispiel HTTP Put
  \end{itemize}
\end{frame}

\begin{frame}
  \frametitle{DHT}
  \framesubtitle{Eigenschaft}
  \begin{itemize}
    \item Schlüssel-Wert-Speichersystem
    \item Bedeutung im Peer-to-Peer-Netzwerk
    \item Wichtig ist der Schlüsselraum
  \end{itemize}
\end{frame}