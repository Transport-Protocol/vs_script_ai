%\section{Einleitung}
\subsection{Dependency Injection}
\begin{frame}
  \frametitle{Dependency Injection}
  \framesubtitle{Idee}
  \begin{itemize}
    \item Abhängigkeiten eines Objekts wird von außen bereitgestellt
    \item Code wird modularer, testbarer und wartbarer
    \item Gleichen Vorteile wie das Factory Pattern
    \item Führt zusätzliche Komplexität ein 
    \item Leistung kann reduziert werden, wenn Abhängigkeiten über das Netzwerk transportiert werden muss
    \item Kann Sicherheitsprobleme einführen
    \item Problematisch im Debugging
  \end{itemize}
\end{frame}

\begin{frame}[fragile]
  \frametitle{Dependency Injection}
  \framesubtitle{Umsetzung in Code I}
  \begin{minipage}{\textwidth}
  \begin{lstlisting}[caption={Schnittstelle für die Abhängigkeit},captionpos=b,label={lst:di-interface}]
  public interface BulbService {
      void doLight();
  }
  \end{lstlisting}
  \end{minipage}
\end{frame}

\begin{frame}[fragile]
  \frametitle{Dependency Injection}
  \framesubtitle{Umsetzung in Code II}
  \begin{minipage}{\textwidth}
  \begin{lstlisting}[caption={Schnittstellenimplementierung},captionpos=b,label={lst:di-interface-implementation}]

  public class Bulb implements BulbService {
      public void doLight() {
          System.out.println("Do cool Stuff");
      }
  }
  \end{lstlisting}
  \end{minipage}
\end{frame}

\begin{frame}[fragile]
  \frametitle{Dependency Injection}
  \framesubtitle{Umsetzung in Code III}
  \begin{minipage}{\textwidth}
  \begin{lstlisting}[caption={Dependency},captionpos=b,label={lst:di-dependency}]

  public class ControllerOnPI {
      private BulbService service;
      
      public ControllerOnPI(BulbService service) {
          this.service = service;
      }
      
      public void doControl() {
          service.doLight();
      }
  }
  \end{lstlisting}
  \end{minipage}
\end{frame}

\begin{frame}[fragile]
  \frametitle{Dependency Injection}
  \framesubtitle{Umsetzung in Code IV}
  \noindent\begin{minipage}{\textwidth}
  \begin{lstlisting}[caption={Die DI Main},captionpos=b,label={lst:di-main}]
  public class Main {
      public static void main(String[] args) {
          BulbService bulb = new Bulb();
          ControllerOnPI c = new ControllerOnPI(bulb);
          
          c.doControl();
      }
  }
  \end{lstlisting}
  \end{minipage}
\end{frame}

\begin{frame}
  \frametitle{Dependency Injection}
  \framesubtitle{Reflection}
  \begin{itemize}
    \item Dependency Injection (DI) wird häufig mit Reflection umgesetzt
    \item Struktur und Verhalten zur Laufzeit analysieren
    \item Eigenschaften und Funktionalitäten können zur Laufzeit verändert werden
    \item In vielen Programmiersprachen unterstützt, z. B. in Java, C\#, Python und JavaScript
  \end{itemize}
\end{frame}