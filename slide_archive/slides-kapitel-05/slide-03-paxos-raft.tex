
\subsection{Konsens}
\begin{frame}
  \frametitle{Algorithmen}
  \framesubtitle{Konsens}
  \begin{itemize}
    \item Viele Systeme auf hochverfügbare und korrekte Systeme angewiesen
    \item Einzelne Kerne, CPUs oder Systeme dem nicht gewachsen
    \item Redundanz hilfreich? Zwingend Zusammenarbeit/ Kooperation!
  \end{itemize}
\end{frame}


\begin{frame}
  \frametitle{Konsens}
  \framesubtitle{Problem}
  \begin{itemize}
    \item Es existiert eine Gruppe 
    \item Teilnehmer kommunizieren mit Nachrichten
    \item Teilnehmer oder Nachrichten können fehlerhaft sein
  \end{itemize}
\end{frame}

\begin{frame}
  \frametitle{Konsens Fallbeispiel}
  \framesubtitle{Anwendungsbeispiel}
  \begin{itemize}
  \item Verteiltes System von zwei Ampeln, das den Verkehr an einer Baustelle reguliert
  \item Jede Ampel arbeitet autonom 
  \item Koordiniert sich mit anderer Ampel über Nachrichten
  \end{itemize}
\end{frame}

\begin{frame}
  \frametitle{Konsens}
  \framesubtitle{Anforderungen}
  \begin{itemize}
    \item Einigkeit
    \item Integrität
    \item Terminierung
    \item Validität
  \end{itemize}
\end{frame}


\begin{frame}
  \frametitle{Konsens}
  \framesubtitle{Follow the Leader}
  \begin{itemize}
    \item Ein Knoten wird zu Leader gewählt
    \item Reduziert Komplexität
    \item Paxos nutzt alternativen Multi-Leader Ansatz
    \item P2P Architekturen können auch Leaderless funktionieren
  \end{itemize}
\end{frame}


\begin{frame}
  \frametitle{Konsens}
  \framesubtitle{Bekannte Lösungen}
  \begin{itemize}
    \item Paxos (Mutter)
    \item RAFT (Einfacher)
    \item ZAB (Zookeeper)
  \end{itemize}
\end{frame}

\begin{frame}
  \frametitle{Konsens}
  \framesubtitle{Paxos}
  \begin{itemize}
    \item Leslie Lamport 1990
    \item Mutter aller Konsens-Algorithmen
    \item Arbeitet mit Rollen: Proposer oder Acceptors
    \item Berüchtigt für seine Komplexität
  \end{itemize}
  \footnote{Weitere Details im Script}
\end{frame}

\begin{frame}
  \frametitle{Konsens}
  \framesubtitle{Raft}
  \begin{itemize}
    \item Diego Ongaro und John Ousterhout (2013/2014)
    \item Alternative zu komplexen Paxos
    \item Konzentriert sich auf drei Aufgaben
    \begin{itemize}
      \item Leader Election
      \item Log Replication  
      \item Safety
    \end{itemize}
  \end{itemize}
  \footnote{Weitere Details im Script}
\end{frame}

\begin{frame}
  \frametitle{Konsens}
  \framesubtitle{ZAB}
  \begin{itemize}
    \item Speziell für ZooKeeper entworfen und optimiert
    \item Zuverlässige und geordnete Zustandsaktualisierung in einer Replikationsgruppe
    \item Im Wesentlichen ein Zwei-Phasen-Commit-Protokoll mit einer Führungsfindungsphase
    \item Toll für Namensdienst, Konfigurationsmanagement, etc
    \item Aber schlecht auf großen Datenmengen, bei vielen Schreibvorgängen, etc 
  \end{itemize}
    \footnote{Weitere Details im Script}
\end{frame}


\begin{frame}
  \frametitle{Konsens}
  \framesubtitle{Konsens mit Crash Failures}
  \begin{itemize}
    \item Unmöglichkeit von 1-Crash-Konsens (fundamentale Beweis 1985)
    \item Als FLP-Unmöglichkeitstheorem bezeichnet
    \item Unterscheidung Nachricht verzögert oder Empfänger abgestürzt
  \end{itemize}
  \textcolor{red}{Für ein asynchrones verteiltes System mit nur einem möglicherweise fehlerhaften Prozess (d.h. einem Prozess, der abstürzen kann) kann es keinen deterministischen Algorithmus geben, der immer zu einem Konsens führt}\footnote{Ein asynchrones System ist ein System, in dem es keine festen Obergrenzen für die Zeit gibt}
\end{frame}

\begin{frame}
  \frametitle{Konsens}
  \framesubtitle{FLP-Unmöglichkeitstheorem}
  \begin{itemize}
    \item Teilsynchrone Systeme (Asynchron ist kaum durchzuhalten)
    \item Randomisierte Algorithmen (Randomized Consensus Algorithmen-Gruppe)
    \item Praktische Konsensalgorithmen (Paxos, Raft oder ZAB)
    \item Konsens durch teilweise Synchronität
    \begin{itemize}
      \item Bracha-Toueg Crash Consensus
      \item Chandra-Toueg Unreliable Failure Detectors 
    \end{itemize}
  \end{itemize}
\end{frame}

\begin{frame}
  \frametitle{Konsens}
  \framesubtitle{Unreliable Failure Detectors }
  \begin{itemize}
    \item Fehlerdetektoren müssen nicht perfekt sein (False positives)
    \item Beispiel ist: Timeout-basierter Fehlerdetektoren
    \item Alternativen sind Heartbeat-Mechanismus, Quorum-basierte Ansätze
  \end{itemize}
\end{frame}

\begin{frame}
  \frametitle{Konsens}
  \framesubtitle{Konsens mit Byzantine Failures}
  \begin{itemize}
    \item Basis weiter Byzantinische Generäle Problem
    \item Mehr als zwei Drittel der Knoten müssen korrekt sein
    \item Möglicher algorithmischer Ansatz Practical Byzantine Fault Tolerance (PBFT)
  \end{itemize}
\end{frame}

