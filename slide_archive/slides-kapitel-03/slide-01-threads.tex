\subsection{Threads}
\begin{frame}
  \frametitle{Threads}
  \framesubtitle{Motivation für VS}
  \begin{itemize}
    \item Parallelität
    \item Effiziente Ressourcennutzung
    \item Reaktionsfähigkeit
    \item Einfachere Kommunikation und Synchronisation 
    \item Granularität
  \end{itemize}
\end{frame}

\begin{frame}
  \frametitle{Threads}
  \framesubtitle{Herausforderung für VS}
  \begin{itemize}
    \item Deadlocks
    \item Race Conditions
    \item Verteilung/Skalierung über nodes
  \end{itemize}
\end{frame}

\begin{frame}
  \frametitle{Threads}
  \framesubtitle{Thread Modelle}
  \begin{itemize}
    \item Many-to-One (User Thread)
    \item One-to-One-Thread-Modell (Kernel Thread)
    \item N:M-Thread-Modell (Nicht so üblich)
  \end{itemize}
\end{frame}

\begin{frame}
  \frametitle{Threads}
  \framesubtitle{VS Anwendung Threadpool}
  \begin{itemize}
    \item Effiziente Ressourcennutzung
    \item Einfache Verwaltung
  \end{itemize}
\end{frame}

\begin{frame}[fragile]
  \frametitle{Threads}
  \framesubtitle{VS Anwendung Threadpool}
\begin{lstlisting}[caption={ExecutorService-Klasse I},captionpos=b,label={lst:executorI}]

import java.util.concurrent.ExecutorService;
import java.util.concurrent.Executors;

public class ThreadPoolExample {
    public static void main(String[] args) {
        int numThreads = 5;
        ExecutorService executor = Executors.newFixedThreadPool(numThreads);

        for (int i = 0; i < 10; i++) {
            Runnable task = new ExampleTask(i);
            executor.execute(task);
        }

        executor.shutdown();
    }
}
\end{lstlisting}
\end{frame}

\begin{frame}[fragile]
  \frametitle{Threads}
  \framesubtitle{VS Anwendung Threadpool}
  \begin{lstlisting}[caption={ExecutorService-Klasse II},captionpos=b,label={lst:executorII}]

  import java.util.concurrent.ExecutorService;
  import java.util.concurrent.Executors;

  public class ThreadPoolExample {
      public static void main(String[] args) {
          int numThreads = 5;
          ExecutorService executor = Executors.newFixedThreadPool(numThreads);

          for (int i = 0; i < 10; i++) {
              Runnable task = new ExampleTask(i);
              executor.execute(task);
          }

          executor.shutdown();
      }
  }

\end{lstlisting}
\end{frame}

\begin{frame}[fragile]
  \frametitle{Threads}
  \framesubtitle{VS Anwendung Threadpool}
  \begin{lstlisting}[caption={ExecutorService-Klasse III},captionpos=b,label={lst:executorIII}]  
  class ExampleTask implements Runnable {
      private int taskId;

      public ExampleTask(int taskId) {
          this.taskId = taskId;
      }

      @Override
      public void run() {
          System.out.println("Task " + taskId + " is running on thread: " + Thread.currentThread().getName());
      }
  }
\end{lstlisting}
\end{frame}

\begin{frame}
  \frametitle{Threads}
  \framesubtitle{VS Anwendung Threadpool}
  \begin{itemize}
    \item Im besten Fall Umsetzung mit Tasks und Interface Strukturen 
    \item Anzahl Threads maximal gleich der Anzahl der verfügbaren Prozessorkerne bei rechenintensiven Aufgaben
    \item Mehr Threads als Prozessorkerne nur sinnvoll bei blockierenden Strukturen
  \end{itemize}
\end{frame}

\begin{frame}
  \frametitle{Threads}
  \framesubtitle{Threadpool Task}
  \begin{itemize}
    \item Modularität 
    \item Kapselung
    \item Idempotenz
    \item Zustandslosigkeit
    \item Fehlertoleranz
    \item Skalierbarkeit
    \item Kommunikation
  \end{itemize}
\end{frame}

\begin{frame}
  \frametitle{Threads}
  \framesubtitle{Multithreaded Clients}
  \begin{itemize}
    \item Oft mit verschiedenen Arten von Aufgaben konfrontiert
    \item Aufgaben unabhängig voneinander und gleichzeitig
    \item Verwendung von Threads typischerweise zur Organisation
  \end{itemize}
\end{frame}

\begin{frame}
  \frametitle{Threads}
  \framesubtitle{Multithreaded Clients}
  \begin{itemize}
    \item Reaktionsfähigkeit
    \item Modularität und Wartbarkeit
    \item Bessere Ressourcennutzung
    \item Vereinfachte Kommunikation
    \item Beispiel Browser: Surfen und gleichzeitiger Datei-Download
  \end{itemize}
\end{frame}

\begin{frame}
  \frametitle{Threads}
  \framesubtitle{Thread-Level-Parallelism (TLP)}
  \begin{itemize}
    \item Maß dafür, wie viele Threads in einer Anwendung gleichzeitig ausgeführt werden können
    \item Berechnung basiert auf Speedup
  \end{itemize}
  \begin{equation}
    TLP = \frac{Speedup}{p} = \frac{\frac{T_1}{T_p}}{p}
  \end{equation}
  \begin{itemize}
    \item Beispiel TLP von 1,5 -2 bedeuted 1,5 bis 2 Threads gleichzeitig auszuführen.
  \end{itemize}
\end{frame}


\begin{frame}
  \frametitle{Threads}
  \framesubtitle{Single Threaded Process}
  \begin{itemize}
    \item Nur ein einzigen Ausführungsstrang (Thread)
    \item Eine Aufgabe zur Zeit abgearbeitet
    \item Daher sequenziellen Reihenfolge der Aufgaben
    \item Beispiel nodejs
  \end{itemize}
\end{frame}


\begin{frame}
  \frametitle{Threads}
  \framesubtitle{Single Threaded Process}
  \begin{itemize}
    \item Einfachheit
    \item Skalierbarkeit
    \item Geringerer Ressourcenverbrauch
    \item Nutzt Rechenleistung von Mehrkernprozessoren nicht voll aus
    \item Gut im Cluster-Betrieb für Single-Node Architekturen
    \item Notwendig: Blocking mit Timeout oder Non-Blocking IO
  \end{itemize}
\end{frame}


\begin{frame}[fragile]
  \frametitle{Threads}
  \framesubtitle{Single Threaded Process}
  \begin{lstlisting}[caption={Node.js Single Threaded},captionpos=b,label={lst:single}]
  const http = require("http");

  const server = http.createServer((req, res) => {
    console.log("Anfrage empfangen");

    // Simuliere eine zeitaufwaendige Operation
    setTimeout(() => {
      res.writeHead(200, { "Content-Type": "text/plain" });
      res.end("Hallo Welt!");
    }, 1000);
  });

  server.listen(3000, () => {
    console.log("Server laeuft auf Port 3000");
  });
  \end{lstlisting}
\end{frame}

\begin{frame}
  \frametitle{Threads}
  \framesubtitle{Non-Blocking I/O}
  \begin{itemize}
    \item Asynchrone I/O (AIO)
    \item I/O-Multiplexing (auch bekannt als Event-Driven I/O)
    \item Non-Blocking Sockets
    \item Beispiel für NIO ist java.nio.channels (Beispiel Script)
  \end{itemize}
\end{frame}

\begin{frame}
  \frametitle{Threads}
  \framesubtitle{Server}
  \begin{itemize}
    \item Iterativ
    \item Nebenläufig
    \item Beispielimplementierungen im Script
  \end{itemize}
\end{frame}

\begin{frame}
  \frametitle{Threads}
  \framesubtitle{Iterativer Server}
  \begin{itemize}
    \item Einfache Implementierung
    \item Geringerer Ressourcenverbrauch
    \item Herausforderung: Skalierbarkeit über Kerne
    \item Herausforderung: Reaktionsfähigkeit
  \end{itemize}
\end{frame}

\begin{frame}
  \frametitle{Threads}
  \framesubtitle{Nebenläufiger Server}
  \begin{itemize}
    \item Reaktionsfähigkeit
    \item Skalierbarkeit über Kerne
    \item Herausforderung: Komplexität
    \item Herausforderung: Ressourcenverbrauch
  \end{itemize}
\end{frame}
