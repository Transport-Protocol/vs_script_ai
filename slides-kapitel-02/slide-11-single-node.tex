%\section{Einleitung}
\subsection{Single Node (Application0)}
\begin{frame}
  \frametitle{Single Node }
  \framesubtitle{Definition}
  \begin{itemize}
    \item Architekturtyp
    \item Anwendungen werden auf einem einzelnen Knoten ausgeführt
    \item Keine Verteilung der Funktionen
    \item Knoten kann ein physischer Server oder eine virtuelle Maschine sein
    \item Knoten kann von verschiedenen Systemen sammeln und verarbeiten
  \end{itemize}
\end{frame}

\begin{frame}
  \frametitle{Single Node }
  \framesubtitle{Vorteile}
  \begin{itemize}
    \item Entwicklung und Testen
    \item Kleine Anwendungen
    \item Datenintensive Anwendungen
  \end{itemize}
\end{frame}

\begin{frame}
  \frametitle{Single Node }
  \framesubtitle{Umsetzung und Beispiele}
  \begin{itemize}
    \item Container-Technologien
    \item Serverless-Computing (AWS Lambda)
    \item Netflix: Netflix verwendet das Single-Node-Pattern, um die Datenverarbeitung in ihren AWS-Cloud-Instanzen
    \item Airbnb: Airbnb verwendet das Single-Node-Pattern, um ihre Webanwendung (nicht die Daten) in einem einzigen Knoten zu hosten
  \end{itemize}
\end{frame}