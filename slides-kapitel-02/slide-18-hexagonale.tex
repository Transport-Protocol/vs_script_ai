%\section{Einleitung}
\subsection{Hexagonal Onion Architektur}
\begin{frame}
  \frametitle{Hexagonal Onion Architektur}
  \framesubtitle{Idee}
  \begin{itemize}
    \item Kombination aus zwei bekannten Architekturmustern: Hexagonale Architektur und Zwiebelarchitektur
    \item Die Kernlogik des Systems ist unabhängig von externen Einflüssen 
    \item Ports und Adaptern dienen als Schnittstelle zwischen der Kernlogik und externen Anliegen
  \end{itemize}
\end{frame}

\begin{frame}
  \frametitle{Hexagonal Onion Architecture}
  \framesubtitle{Typische Schichten}
  \begin{itemize}
    \item Domain
    \item Application 
    \item Ports
    \item Adapters
  \end{itemize}
\end{frame}

\begin{frame}
  \frametitle{Hexagonal Onion Architecture}
  \framesubtitle{Beispiel mit Microservices}
  \begin{itemize}
    \item Domain: Verwalten von Produkten und Dienstleister
    \item Application: Prozesse für Kauf, Verkauf, Retoure, Wartung, etc 
    \item Microservice: Online Bestellungs-Service mit Abhängigkeit Bezahlservice
    \item Microservice: Online Produkt-Service
    \item Microservice: Online Zahlungs-Service
    \item Microservice: Online Versand-Service 
  \end{itemize}
\end{frame}