\section{Veranstaltung VS}
\subsection{Anforderungen}
\begin{frame}
  \frametitle{Anforderungen für Einstieg VS}
  \framesubtitle{Was sollte ich mitbringen? }
  \begin{itemize}
    \item Interesse daran, wie verteilte Anwendungen funktionieren
    \item Ein wenig Englisch ist hilfreich
    \item Ausreichende Kenntnisse der Programmiersprachen Java oder C++
    \item Kenntnisse der Module: AD, DB, Programmier*, SE, BS, RN 
  \end{itemize}
\end{frame}
\subsection{Resourcen}
\begin{frame}
  \frametitle{Resourcen}
  \framesubtitle{Wo finde ich was?}
  \begin{itemize}
    \item Zentraler Punkt: MS Teams (Code: jt4n0ed)
    \item Zentrales Script: NEW Version 0.9 (Im Aufbau)
    \item Referenzliteratur und Basis früherer Vorlesungen \url{https://www.distributed-systems.net/index.php/books/ds3/}
    \item Folien mit der Struktur der Vorlesung 
  \end{itemize}
\end{frame}

\subsection{Prüfung}
\begin{frame}
  \frametitle{Prüfungsform}
  \framesubtitle{Abhängig der Teilnehmer}
  \begin{itemize}
    \item Default: Klausur
    \item 90 Minuten 
    \item Alles Klausurrelevant (Unterlagen, Vorlesung, Praktikum)
    \item Aufgaben stark mit Fokus auf Transfer
  \end{itemize}
\end{frame}

\subsection{Praktikum}
\begin{frame}
  \frametitle{Praktikumsinformationen}
  \framesubtitle{Entwicklungszyklus in VS}
  \begin{itemize}
    \item Standalone Applikation
    \item Client/-Server mit Middleware
    \item RPC-Architektur 
    \item Verteilter Algorithmus
  \end{itemize}
\end{frame}

\subsection{Praktikum - Best Practice}
\begin{frame}
  \frametitle{Praktikumshinweise}
  \framesubtitle{Was sollte ich beachten?}
  \begin{itemize}
    \item Maximale Gruppengröße 3 (auch 2 oder 1 möglich).
    \item Praktikum ist zeitaufwendig 
    \item Praktikum ist ein iterativer Prozess 
    \item Praktikum hat keine Anwesenheitspflicht, nur ein Abtestat von Milestones
  \end{itemize}
\end{frame}