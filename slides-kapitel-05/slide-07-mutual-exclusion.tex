%\section{Einleitung}
\subsection{Mutual Exclusion}
\begin{frame}
  \frametitle{Algorithmen}
  \framesubtitle{Mutual Exclusion}
  \begin{itemize}
    \item Mutual Exclusion, oder wechselseitiger Ausschluss zum Schutz von kritischen Bereichen
    \item Notwendig in VS zur Sicherung der Datenkonsistenz und -integrität in Mehrprozess- oder Mehrfadenanwendungen
  \end{itemize}
\end{frame}

\begin{frame}
  \frametitle{Mutual Exclusion}
  \framesubtitle{Ansätze}
  \begin{itemize}
    \item Ricart-Agrawala-Algorithmus (Hohe Kommunikation)
    \item Raymond's Tree-Based Algorithmus (Hohe Netzstabilität)
    \item Agrawal-El Abbadi-Algorithmus (Schnelle und stabile Netze)
    \item Peterson's Algorithmus (Akademischer Ansatz zur Lehre)
    \item Bakery Algorithmus (Problem bei vielen Teilnehmern)
    \item Fischer's Test-Set-and-Set-Lock Algorithmus (Bei geringen Einsätzen)
    \item  Queue-Locks (Lamport Clocks, Multicast mit Acks)
  \end{itemize}
\end{frame}

\subsection{Barriers}
\begin{frame}
  \frametitle{Algorithmen}
  \framesubtitle{Mutual Exclusion}
  Klassisches Warten auf andere Ablauffäden
  \begin{itemize}
    \item Sense-Reversing Barrier (Einfacher Counter)
    \item Combining Tree Barrier  (Baumstruktur)
    \item Tournament Barrier      (Unstrukturiert mit Paaren)
    \item Dissemination Barriers  (Gossip-ähnlich, Runden-basiert)
  \end{itemize}
\end{frame}