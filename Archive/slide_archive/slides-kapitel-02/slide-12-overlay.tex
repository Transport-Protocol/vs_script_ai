%\section{Einleitung}
\subsection{Overlay Network}
\begin{frame}
  \frametitle{Overlay}
  \framesubtitle{Idee}
  \begin{itemize}
    \item Logische Verbindung muss keiner physikalischen Verbindung entsprechen
    \item Genutzte Netzwerkstruktur besteht häufig auf der obersten Schicht einer physikalischen Netzwerkstruktur
    \item Zusätzliche Abstraktion und Organisation
    \item Zusätzliche Komplexität und Overhead
  \end{itemize}
\end{frame}

\begin{frame}
  \frametitle{Overlay}
  \framesubtitle{Beispiel P2P}
  \begin{itemize}
    \item In einem P2P-Netzwerk sind die Knoten gleichberechtigte Teilnehmer
    \item Teil oder voll vermascht 
    \item Bekanntes Beispiel für ein P2P-Overlay-Netzwerk ist BitTorrent
  \end{itemize}
\end{frame}

\begin{frame}
  \frametitle{Overlay}
  \framesubtitle{P2P - Strukturierte Netzwerke}
  \begin{itemize}
    \item Geordnete Struktur
    \item Verwenden meist deterministische Verfahren oder Algorithmen
    \item Routing effizienter und vorhersagbarer 
    \item Ein Beispiel für ein strukturiertes Routing-Verfahren ist DHT
    \item Suche kann in logarithmischer Zeit (O(log N)) durchgeführt werden
  \end{itemize}
\end{frame}

\begin{frame}
  \frametitle{Overlay}
  \framesubtitle{P2P - Un-Strukturierte Netzwerke}
  \begin{itemize}
    \item Keine Struktur
    \item Verwendung auf Basis von Heuristiken und Zufall
    \item Routing kaum effizient zu gestalten 
    \item Typisch ist hohe Netzwerklast
    \item Flooding- oder Random-Walk-Verfahren für die Suche
    \item Schädliche Teilnehmer ein besonderes Problem
  \end{itemize}
\end{frame}