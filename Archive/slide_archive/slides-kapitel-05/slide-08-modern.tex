%\section{Einleitung}
\subsection{Self-Stabilization}
\begin{frame}
  \frametitle{Algorithmen}
  \framesubtitle{Self-Stabilization}
  \begin{itemize}
    \item Ansatz der Fehlertoleranz für Algorithmen
    \item Aus beliebigen initialen Zuständen in einen korrekten Zustand zu wechseln
    \item Wichtig es garantiert nach einer endlichen Anzahl von Schritten einen korrekten Zustand
  \end{itemize}
\end{frame}

\begin{frame}
  \frametitle{Self-Stabilization}
  \framesubtitle{Ansätze}
  \begin{itemize}
    \item Dijkstra's Token Ring for Mutual Exclusion (Garantiert Existenz von einem Token)
    \item Arora-Gouda Spanning Tree (Kann beliebig neue Bäume erstellen)
    \item Afek-Kutten-Yung Spanning Tree (Verbesserung von  Arora-Gouda Spanning Tree - Gruppen von Knoten)
  \end{itemize}
\end{frame}

\subsection{Smart Contracts}
\begin{frame}
  \frametitle{Algorithmen}
  \framesubtitle{Smart Contracts}
  \begin{itemize}
    \item \enquote{Distributed Ledger Technology} (DLT)  => Zentrale Datenbasis
    \item \enquote{Ledger} ist im Grunde eine Liste von (digitalen) Transaktionen
    \item Bekanntes Beispiel Blockchain
  \end{itemize}
\end{frame}

\begin{frame}
  \frametitle{Smart Contracts}
  \framesubtitle{Blockchain}
  \begin{itemize}
    \item Blockchains sind eine Form der verteilten Ledger-Technologie
    \item Protokollierungen von Transaktionen in einem Peer-to-Peer-Netzwerk
    \item Jede Transaktion wird in einem Block aufgezeichnet
    \item Jeder Block ist mit dem vorhergehenden Block verbunden
    \item Ergibt eine Kette von Blöcken oder eine \enquote{Blockchain}
    \item Jede Transaktion kann von jedem Knoten im Netzwerk gesehen und verifiziert werden
    \item Das Verbinden wird mit kryptographischen Verfahren abgesichert
  \end{itemize}
\end{frame}

\begin{frame}
  \frametitle{Smart Contracts}
  \framesubtitle{Idee}
  \begin{itemize}
    \item Smart Contracts sind Programme, die auf einer Blockchain ausgeführt werden
    \item Sie sind \enquote{smart}, weil sie automatisch ausgeführt werden
    \item Sobald ein Smart Contract auf die Blockchain hochgeladen wurde, kann er nicht mehr geändert oder gestoppt werden (Dezentrale Börse)
    \item Ethereum ist eine andere Open-Source-Blockchain-Plattform 
    \item  Aktuell viele dezentrale Anwendungen (dApps), die auf Smart Contracts basieren (Beispiel: Wettsysteme, Logistik, Spiele, etc)
    \item Jede Transaktion kann von jedem Knoten im Netzwerk gesehen und verifiziert werden
  \end{itemize}
\end{frame}