%\section{Einleitung}
\subsection{Paralleles Rechnen}
\begin{frame}
  \frametitle{Paralleles Rechnen}
  \framesubtitle{Parallele Algorithmen}
  \begin{itemize}
    \item Leicht: \enquote{embarrassingly parallel} 
    \begin{itemize}
      \item Matrixoperationen
      \item Sortieralgorithmen
      \item Grafikverarbeitung
      \item ...
    \end{itemize}
    \item Unterstützt durch Frameworks (z.b. MapsReduce\footnote{Siehe Script})
  \end{itemize}
\end{frame}


\begin{frame}
  \frametitle{Paralleles Rechnen}
  \framesubtitle{Parallele Algorithmen}
  \begin{itemize}
    \item Nicht alle Algorithmen gut für die Parallelisierung geeignet
    \item Beispiel ggT
    \item Strategie kann sein Granularität zu ändern
  \end{itemize}
\end{frame}

\begin{frame}
  \frametitle{Algorithmenstrukturn}
  \framesubtitle{Koordination}
  \begin{itemize}
      \item Leistung
      \item Zuverlässigkeit
      \item Konsistenz
      \item Sicherheit
  \end{itemize}
  One-to-Many-Kommunikation, Baumstrukturen, Flooding oder Gossip: 
  \begin{itemize}
    \item Zentralisierte Algorithmen 
    \item Dezentralisierte Algorithmen
  \end{itemize}
\end{frame}