%\section{Einleitung}
\subsection{Cluster und Grid}
\begin{frame}
  \frametitle{Cluster}
  \framesubtitle{Einleitung}
  \begin{itemize}
    \item Gruppe von Computern oder Servern
    \item Hochverfügbarkeit, Fehlertoleranz und Leistungsverbesserung
    \item In der Regel homogene HW
    \item Bietet einheitliche Schnittstelle
    \item Einsatz: High-Performance-Computing, Web-Hosting, Datenbank-Management
    \item Arten (Grundlage): Beowulf und Wolfpack 
  \end{itemize}
\end{frame}

\begin{frame}
  \frametitle{Cluster}
  \framesubtitle{Beowulf}
  \begin{itemize}
    \item 1990er Jahren von Thomas Sterling und Donald Becker entwickelt
    \item Handelsüblicher Hardware und Open-Source-Software
    \item Hauptanwendung HPC
    \item Idee als Konkurrenz zu Großrechner
    \item Nutzen z.B. Message Passing Interface (MPI) oder Parallel Virtual Machine (PVM)
    \item kosteneffizient
  \end{itemize}
\end{frame}

\begin{frame}
  \frametitle{Cluster}
  \framesubtitle{Wolfpack}
  \begin{itemize}
    \item 1990er Jahren ursprünglich von Microsoft
    \item Teil der Windows NT Server-Produktlinie
    \item Wolfpack bietet Hochverfügbarkeit und Fehlertoleranz
  \end{itemize}
\end{frame}

\begin{frame}
  \frametitle{Cluster}
  \framesubtitle{InfiniBand }
  \begin{itemize}
    \item Konkurenz zu klassichen Ethernet Strukturen
    \item Geringere Latenz und hohe Bandbreit (bis zu mehreren hundert Gigabit pro Sekunde)
    \item Von der InfiniBand Trade Association (IBTA) als Standard definiert
    \item Basis ist serielles, punkt-zu-punkt Kommunikationsprotokoll
    \item Beispiel für Einsatz: Frankfurter Börse
  \end{itemize}
\end{frame}

\begin{frame}
  \frametitle{Grid}
  \framesubtitle{Einleitung }
  \begin{itemize}
    \item Geografisch verteilte und heterogene Computerressourcen
    \item Unterschiedlichen Hardware- und Softwarekonfigurationen
    \item Ressourcen in der Regel autonom
  \end{itemize}
\end{frame}