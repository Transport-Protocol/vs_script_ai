\section{Allgemeine VS Anforderungen}
%\subsection{Einleitung}
\begin{frame}
  \frametitle{Allgemeine VS Anforderungen}
  \framesubtitle{Anforderungen Teil eines iterativen Prozess}
  \begin{itemize}
    \item Anforderungen (Funktionale und Nicht-funktionale)
    \item Analyse und Design (Architektur)
    \item Implementierung
    \item Test
    \item Deployment
  \end{itemize}
\end{frame}



\begin{frame}
  \framesubtitle{Mögliche Schritte für ein gutes Design}
  \begin{itemize}
    \item Definieren der Ziele und Anforderungen
    \item Anforderungen erfassen
    \item Anforderungen analysieren und spezifizieren
    \item Validieren und verifizieren der Anforderungen
    \item Verwalten der Anforderungen
  \end{itemize}
\end{frame}


\begin{frame}
  \framesubtitle{Auswahl von Methoden}
  \begin{itemize}
    \item Anwendungsfalldiagramme
    \item Datenflussdiagramme
    \item Architekturdiagramme
    \item Qualitätsattribute und Szenarien
    \item Interview- und Workshop-Methoden
  \end{itemize}
  Vieles hängt an einer angemessenen Dokumentation\footnote{Im Praktikum wird ARC42 als Dokumentationsmethode angeboten: \url{https://arc42.org/download}}
\end{frame}
